% Abstract

\begin{abstract}

	Una presentación electrónica es un conjunto de diapositivas que a su vez, se visualizan en el monitor de una computadora, o bien se proyectan en una pantalla a distancia. En la actualidad, existen diversas herramientas que permiten crear y controlar diapositivas: PowerPoint, Beamer, Impress de LibreOffice, Keynote para Mac, entre muchas. También existen otras en ámbito web como son: Impress.js, Reveal.js, jmpress.js, Prezi. Cada una de ellas tiene ventajas y desventajas, y no existe ninguna que reúna las mejores características de todas.


	El objetivo de este trabajo es diseñar una aplicación web que permita mostrar y controlar diapositivas creadas con HTML5, CSS3 y JavaScript y que simule todas las funcionalidades de Beamer, teniendo una estructura de sintaxis similar a dicho paquete. Las presentaciones ofrecen la posibilidad de tener varias vistas disjuntas, así como un panel de control para administrar el progreso de la visualización.

\end{abstract}
