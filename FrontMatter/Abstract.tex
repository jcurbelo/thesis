% Abstract

\begin{abstract}

	Una presentación electrónica es un conjunto de diapositivas que a su vez, se visualizan en el monitor de una computadora, o bien se proyectan en una pantalla a distancia. En la actualidad, existen diversas herramientas que permiten crear y controlar diapositivas: PowerPoint, Beamer, Impress de LibreOffice, Keynote para Mac, entre muchas. También existen otras en ámbito web como son: Impress.js, Reveal.js, jmpress.js, Prezi. Cada una de ellas tiene ventajas y desventajas, y no existe ninguna que reúna las mejores características de todas.


	El objetivo de este trabajo es diseñar una aplicación web que permita mostrar y controlar diapositivas creadas con HTML5, CSS3 y JavaScript y que simule todas las funcionalidades de Beamer, teniendo una estructura de sintaxis similar a dicho paquete. Las presentaciones ofrecen la posibilidad de tener varias vistas disjuntas, así como un panel de control para administrar el progreso de la visualización.

	% An electronic presentation is a set of slides which are displayed on a computer monitor or projected on a screen at a distance. Currently, there are various tools that create and control slides: PowerPoint, Beamer, Impress, Keynote for Mac among many others. Also there are other web based tools including: Impress.js, Reveal.js, jmpress.js, Prezi. Each has advantages and disadvantages, and there is none that brings the best features of all.


	% The aim of this work is to design a web application that can display and control slides created using HTML5, CSS3 and Javascript and simulating all the features of Beamer, having a structure syntax similar to that package. The presentations offer the ability to have multiple disjoint views, so as a control panel to manage the progress of the visualization.

\end{abstract}
