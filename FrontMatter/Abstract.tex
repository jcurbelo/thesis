% Abstract

\begin{abstract}

Una presentación electrónica es un conjunto de diapositivas que a su vez, se visualizan en el monitor de una computadora, o bien se proyectan en una pantalla a distancia. En la actualidad, existen diversas herramientas que permiten crear y controlar diapositivas: PowerPoint, Beamer, Impress de LibreOffice, Keynote para Mac entre muchas otras. También existen otras basadas en Html, CSS y Javascript, como son: Impress.js http://bartaz.github.io/impress.js/, Reveal.js http://lab.hakim.se/reveal-js/,  jmpress.js http://jmpressjs.github.io/jmpress.js/, Prezi.
El objetivo de este trabajo es diseñar una aplicación web que permita mostrar y controlar diapositivas creadas con HTML5, CSS3 y Javascript y que simule todas las funcionalidades de Beamer, teniendo una estructura de sintaxis similar a dicho paquete. Las presentaciones ofrecen la posibilidad de tener varias vistas disjuntas, así como un panel de control para administrar el progreso de la visualización.


\end{abstract}
