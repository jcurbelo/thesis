% Abstract

\begin{abstract}

Una presentación electrónica se puede ver un conjunto de diapositivas que pueden visualizarse en el monitor de la computadora o bien, proyectarse sobre una pantalla. En la actualidad existen diversas herramientas que permiten crear y controlar diapositivas: PowerPoint, beamer, Impress de LibreOffice, Keynote para Mac entre muchas otras. También existen otras basadas en Html, CSS y Javascript, como son: Impress.js http://bartaz.github.io/impress.js/, Reveal.js http://lab.hakim.se/reveal-js/,  jmpress.js http://jmpressjs.github.io/jmpress.js/, Prezi etc.
El objetivo de este trabajo es diseñar una aplicación web que permita mostrar y controlar diapositivas creadas con HTML5, CSS3 y javascript y que simule todas las funcionalidades de Beamer teniendo una estructura de sintaxis simlar a dicho paquete. Las presentaciones pueden ser extendidas con animaciones efecutadas en javascript, contarán con la capacidad de tener varias vistas disjuntas y un panel de control para administrar el progreso de la presentación.


\end{abstract}
