% SupervisorOpinion

\begin{opinion}

    Las presentaciones en público son importantes, y una prueba de ello es la cantidad de aplicaciones existentes para diseñar las diapositivas.  Las más conocidas son Powerpoint y la clase Beamer de \LaTeX{}, pero existen otras, incluso más impresionantes que ellas, como impress.js o Reveal.js, basadas en HTML5, CSS, y JavaScript.   

    El trabajo realizado en esta tesis agrega una nueva aplicación a la lista de aplicaciones existentes para mostrar diapositivas en ambiente web.  La característica que la distingue de las existentes es que está diseñada para resolver un problema específico, incorporando algunas de las mejores características de los demás sistemas.

    El problema específico que resuelve esta tesis, es la posibilidad de utilizar en los espectáculos del Proyecto Delta diapositivas que sean ricas desde el punto de vista audiovisual al mismo tiempo que resulten fáciles de diseñar y rápidas de crear.

    Para crear una aplicación con estas características Jorge Curbelo ha tenido que estudiar contenidos que no forman parte de su plan de estudio, identificar y resolver problemas de manera creativa, y lidiar con clientes exigentes, quisquillosos y majaderos.  Creo que todo eso lo ha hecho de una manera excelente, por lo que considero que reúne todos los requisitos para desempeñarse como un excelente científico de la computación.


\vfill
\begin{flushright}
\underline{\hspace{186pt}}\hfill \\
M.Sc. Fernando Ra\'ul Rodríguez Flores
\end{flushright}
\end{opinion}
