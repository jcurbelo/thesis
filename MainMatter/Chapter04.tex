%!TEX root = ../thesis.tex
\chapter{Extensibilidad} % (fold)
\label{cha:extensibilidad}
	Se pueden sobreescribir o añadir nuevas opciones a \textit{beampress.js}; estas son
	las funciones de animación y los índices de los intervalos. Por ejemplo, si se quisiera
	añadir una nueva función se debe hacer lo siguiente, Fig. \ref{fig:ex6}

		\begin{figure}[htb]%
			\begin{lstlisting}%


<script>
   $(document).ready(function(){
       	/**
		 * Las funciones deben tener este formato
		 * @param $el: Selector del elemento HTML  
		 * @param args: Diccionario de los parámetros
		 */
   		function myAnimation($el, args){
   			// Función de animación 
   		}     
       $("#main-container").beampress({
       		"myAnimation": myAnimation
       	});
    });
</script>

			\end{lstlisting}
		\caption{
			Añadiendo una nueva función de animación. 
			\label{fig:ex6} }
		\end{figure}	
% chapter extensibilidad (end)