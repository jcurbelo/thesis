%!TEX root = ../thesis.tex
\chapter{Detalles de implementación} % (fold)
\label{cha:detalles_de_implementacion}
	En este capítulo se explicará cómo fueron implementados \textit{Beampress} y \textit{Beampressk} tomando en cuenta los requerimientos vistos en la sección \ref{sec:requerimientos_basicos_del_sistema_propuesto} y siguiendo las concepciones definidas en el capítulo anterior.

	Cómo se especificó que la presentación tenía que estar en ámbito web, se aprovecharon las bondades de HTML5, CSS3 y Javascript para conformar las diapositivas separando contenido de forma. A continuación se detallan las características del plugin de JQuery \textit{beampress.js} vistas en la sección \ref{sec:beampress} y las implementaciones para las definiciones vistas en \ref{sec:definiciones_de_beampress}. 

	\section{Beampress} % (fold)
	\label{sec:beampress_imp}
	

		 Para la implementación de \textit{beampress.js} se utlizó un patrón ``ligero'' como diseño de dicho plugin \cite{smashingmagazine} que cumple con las siguientes características que son producto del uso de las prácticas más comunes:

		\begin{itemize}
			\item Uso de un punto y coma \textbf{;} antes de llamar la función principal, para evitar conflictos con otros códigos de Javascript que tengan ausencia de dicho signo
			\item Un objeto \textbf{defaults}, que tendrá las configuraciones iniciales del plugin
			\item Un constructor sencillo para la creación y la asignación de la lógica que se desea emplear con el elemento HTML seleccionado
			\item Posibilidad de extender las configuraciones (extender \textbf{defaults})
			\item Un ligero \textit{wrapper} al constructor para evitar problemas como múltiples instancias
		\end{itemize}

		Para crear una presentación usando las definiciones del capítulo anterior se hace uso de los atributos de HTML5. Esto permite estructurar el contenido de las diapositivas usando una sintaxis verbosa, en la fig. \ref{fig:main_container} se especifica donde van definidos los \textbf{frames} con código HTML.

		\begin{figure}[htb]%
			\begin{lstlisting}%

	<div class="main-container">
    	<!-- Los Frames (contenido de la presentación) va aquí -->
	</div>
			\end{lstlisting}
		\caption{Contenedor principal}
		\label{fig:main_container}
		\end{figure}		


		\subsection{Frames} % (fold)
		 \label{sub:frames}
		 
		  
			Un \textit{frame} es un contenedor con la clase \textit{frame}, el orden con que se mostrarán cada uno es el mismo con el que se definan. En la fig. \ref{fig:frames_html} se muestran varios frames.


				\begin{figure}[htb]%
					\begin{lstlisting}%

		<div class="main-container">
    		<section class="frame">Primer Frame</section>
    		<section class="frame">Segundo Frame</section>
    		<!-- ... -->
    		<section class="frame">Enésimo Frame</section>
		</div> 
					\end{lstlisting}
					\caption{Forma de definir frames}
					\label{fig:frames_html}
				\end{figure}		
		% subsection frames (end)

		\subsection{Contenidos} % (fold)
		\label{sub:contenidos}
			Los contenidos de la presentación son cualquier elemento de HTML que se especifique, por ejemplo un \textit{párrafo} (\textit{<p>}), una \textit{imagen} (\textit{<img>}), un \textit{video} (\textit{<video>}), un \textit{audio} (\textit{<audio>}) o un \textit{canvas} (\textit{<canvas>}). En la fig. se definen los contenidos del frame mostrado en la fig. \ref{fig:frame_video_image}.

				\begin{figure}[htb]%
					\begin{lstlisting}%

            <section class="frame">
                <h1>Mostrando varios contenidos</h1>
                <div class="frame-content">
                    <div class="hfil">
                        <div class="minipage">
                            <video controls src="forrest_gump.mp4"></video>
                        </div>
                        <div class="minipage">
                            <img height="280px" src="python.jpg" alt="">
                        </div>                        
                    </div>
                    <div class="block">
                        <header>Estándar</header>
                        <div class="block-content">
                            Bloque estándar
                        </div>
                    </div>
                    <div class="block example">
                        <header>Ejemplo</header>
                        <div class="block-content">
                            Bloque de ejemplo
                        </div>
                    </div>
                    <div class="block alert">
                        <header>Alert</header>
                        <div class="block-content">
                            Bloque alert
                        </div>
                    </div>                    
                </div>
                <footer class="clearfix">
                    <div class="authors">
                        <p>Jorge R. Curbelo Fernández</p>
                    </div>
                    <div class="topic"><p>Mostrando varios contenidos</p></div>
                </footer>                
            </section>
					\end{lstlisting}
					\caption{Un frame en HTML con contenidos}
					\label{fig:frames_html}
				\end{figure}	  
		
		% subsection contenidos (end)

		\subsection{Overlays y Slide Items} % (fold)
		\label{sub:slide_items}
			Como se definió en \ref{def:slide_item}, un \textbf{slide item} es un \textbf{contenido} con un \textbf{conjunto de overlays}, para emplear la sintaxis de overlays vista en la fig. \ref{fig:overlay_set} se hace uso de los atributos \textit{data} de HTML5, en este caso se definió el atributo \textit{data-onslide}. En la fig. \ref{fig:slide_items_html} se ejemplifican varios slide items.

				\begin{figure}[htb]%
					\begin{lstlisting}%

		<div class="block-content">
		    <ul>
		        <li data-onslide="3-">Desde el intervalo 3 hasta el final</li>
		        <li data-onslide="4;6">En el intervalo 4 y 6</li>
		        <li data-onslide="5;8">En el intervalo 5 y 8</li>
		        <li data-onslide="3-5;7-8">Del 3 al 5 y del 7 al 8</li>
		    </ul>          
		</div>	
					\end{lstlisting}
					\caption{Forma de definir slide items con overlays}
					\label{fig:slide_items_html}
				\end{figure}		
		
		% subsection slide_items (end)
		y sus \textit{slide items} serían
		todos los elementos HTML hijos de dicho \textit{frame} como se muestra en la Fig. \ref{fig:ex1}.	 

		Las especificaciones de los intervalos con las animaciones se definen con el atributo \textit{data-onslide}; estas se realizan mediante la combinación de dos sintaxis distintas. La más sencilla, es la misma que se utiliza para definir los \textit{overlays} en Beamer \cite{overlay}, como se ve en la Fig. \ref{fig:ex1}. 
	 

			\begin{figure}[htb]%
				\begin{lstlisting}%

	<section class="frame">
		<h1>Hola, soy el frame 1!!!</h1>
	    <span data-onslide="1-">Desde el intervalo 1 hasta el final.</span>
	    <span data-onslide="2;4">Solamente en los intervalos 2 y 4.</span>
	    <span data-onslide="1;2-4;7-">Desde el 1 hasta el 4 y desde el 7 hasta el final.</span>
	    <span data-onslide="-3;5-8">Desde el 1 hasta el 3 y luego desde el 5 hasta el 8.</span>
	</section>
	<section class="frame">
		<h1>Hola, soy el frame 2!!!</h1>
	</section>
				\end{lstlisting}
			\caption{Ejemplo de un frame con varios overlays y otro mostrando un saludo sin overlays. \label{fig:ex1}}
			\end{figure}

		Con esta sintaxis solo se modifica el valor \textit{opacity} de cada \textit{slide item} en el intervalo que le corresponda, o sea, no tendrán ninguna animación definida. Para esto se cuenta con la otra sintaxis que es un poco más compleja, pero es más explícita sobre lo que se quiere hacer y cuando. Cada intervalo se define mediante un JSON con el siguiente formato Fig. \ref{fig:ex2}:

			\begin{figure}[htb]%
				\begin{lstlisting}%
				<span></span>
	{"slide-number":{"next":{"func":"next-function-name", "args":{...}}, "prev": {"func":"prev-function-name", "args":{...}}}}		
				\end{lstlisting}
			\caption{
				Formato JSON para un intervalo. 
				\label{fig:ex2} }
			\end{figure}

			\begin{itemize}
				\item \textbf{slide-number}: Número del intervalo
				\item \textbf{next-function-name}: Nombre de la función de la transición ``siguiente'' del \textit{slide item}
				\item \textbf{prev-function-name}: Nombre de la función de la transición ``anterior'' del \textit{slide item}			 	
				\item \textbf{args}: Argumentos (en formato de JSON) que recibe la función  
			\end{itemize}


		De esta manera se puede determinar que tipo de función de animación se va a utilizar; beampress.js cuenta con algunas funciones básicas pero el usuario puede utilizar cualquiera que implemente como se verá en la sección \ref{sec:ext}. Nótese el uso de dos funciones: \textit{previous} y \textit{next}, esta es una forma de definir la función inversa de animación de cada \textit{slide item}, que es necesario para las transiciones \textit{previous}. Por ejemplo, si la transición \textit{next} de un \textit{slide item} es ``moverse a la derecha'', tiene sentido que su transición \textit{previous} o función inversa fuera ``moverse a la izquierda''. Nótese que queda a consideración del usuario determinar para cada función su inversa. En la Fig. \ref{fig:ex3} se muestra un ejemplo con la combinación de ambas sintaxis.

			\begin{figure}[htb]%
				\begin{lstlisting}%

	<section class="frame">
		<h1>Usemos ambas sintaxis!!!</h1>
	    <span data-onslide='1-2;{"4":{"next":{"func":"slowShowItem", "args":{}}, "prev": {"func":"slowHideItem", "args":{}}}}'>Desde el intervalo 1 hasta el 2 y luego aparezco en el 4 suavemente.</span>
	    <span data-onslide="2;4">Solamente en los intervalos 2 y 4.</span>
	    <span data-onslide='{"2":{"next":{"func":"slowShowItem", "args":{}}, "prev": {"func":"slowHideItem", "args":{}}}}'>Aparezco suavemente solamente en el intervalo 2.</span>
	    <span data-onslide="-3;5-8">Desde el 1 hasta el 3 y luego desde el 5 hasta el 8.</span>
	</section>				
				\end{lstlisting}
			\caption{
				Ejemplo de uso de ambas sintaxis para definir animaciones con intervalos. 
				\label{fig:ex3} }
			\end{figure}

		Otra ventaja de emplear esta sintaxis es la facilitación de la incorporación de videos y sonidos a la presentación, \textit{beampress.js} viene con funciones básicas para reproducir audio y videos; que estos a su vez se definen en el contenido como se muestra en la Fig. \ref{fig:ex4}

			\begin{figure}[htb]%
				\begin{lstlisting}%

	<section class="frame">
		<h1>Empleo de audio y sonido</h1>
		<video src="example.mp4" data-onslide='{"1":{"next":{"func":"fadeInVideo", "args":{"currentTime": "30"}}, "prev": {"func":"fadeOutVideo", "args":{}}}}'></video>
		<audio src="example.mp3" data-onslide='{"2":{"next":{"func":"fadeInAudio", "args":{"currentTime": "10"}}, "prev": {"func":"fadeOutAudio", "args":{}}}}'></audio>    
	</section>				
				\end{lstlisting}
			\caption{
				Ejemplo de uso de audio y sonido. 
				\label{fig:ex4} }
			\end{figure}		

		Como \textit{beampress.js} está basado en JQuery, para usarlo es necesario incluir alguna biblioteca de JQuery; todo el contenido de la presentación debe estar incluida dentro de un contenedor general, un ejemplo completo de su uso se puede ver en la Fig. \ref{fig:ex5}

			\begin{figure}[htb]%
				\begin{lstlisting}%

	<!DOCTYPE html>
	<html>
		<head>
	    	<title>Ejemplo</title>
		</head>
		<body>
			<div id="main-container">
				<section class="frame">
					<!-- Contenido del frame -->
				</section>
				<section class="frame">
					<!-- Contenido del frame -->
				</section>						
			</div>
		   <script src="js/jquery-1.10.2.min.js"></script>
		   <script src="js/beampress.js"></script>
		   <script>
		       $(document).ready(function(){     
		           $("#main-container").beampress();
		        });
		   </script>		
		</body>
	</html>			
				\end{lstlisting}
			\caption{
				Ejemplo de presentación usando \textit{beampress.js}. 
				\label{fig:ex5} }
			\end{figure}	

		El estilo de la presentación queda a manos del usuario, por lo que se requiere previos conocimientos 
		de HTML, CSS y Javascript, lo cual constituye indudablemente una cierta desventaja. 

	% section beampress (end)
		
% chapter detalles_de_implementacion (end)