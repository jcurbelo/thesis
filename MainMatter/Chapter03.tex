%!TEX root = ../thesis.tex
\chapter{Detalles de implementación} % (fold)
\label{cha:detalles_de_implementacion}
	En este capítulo se explicará cómo fueron implementados \textit{Beampress} y \textit{Beampressk} tomando en cuenta los requerimientos vistos en la sección \ref{sec:requerimientos_basicos_del_sistema_propuesto} y siguiendo las concepciones definidas en el capítulo anterior.

	Cómo se especificó que la presentación tenía que estar en ámbito web, se aprovecharon las bondades de HTML5 y CSS3 para conformar las diapositivas separando contenido de forma. A continuación se explica cómo estructurar una presentación y se detallan las características del plugin de JQuery \textit{beampress.js} que cumplen con las definiciones \ref{def:slide}. 

	\section{Beampress} % (fold)
	\label{sec:beampress_imp}
	

		Como se vió en la sección \ref{sec:beampress} para la implementación de \textit{beampress.js} se utlizó un  \textit{patrón ligero} como diseño de dicho plugin. 

		Para crear una presentación usando las definiciones del capítulo anterior se hace uso de los atributos de HTML5. Esto permite estructurar el contenido de las diapositivas usando una sintaxis semántica, en la fig. \ref{fig:main_container} se especifica donde van definidos los \textbf{frames} con código HTML.


		\subsection{Contenidos} % (fold)
		\label{sub:contenidos}
			Los contenidos de la presentación son cualquier elemento de HTML que se especifique, por ejemplo un \textit{párrafo} (\texttt{<p>}), una \textit{imagen} (\texttt{<img>}), un \textit{video} (\texttt{<video>}), un \textit{audio} (\texttt{<audio>}) o un \textit{canvas} (\texttt{<canvas>}). En la fig. \ref{fig:frames_html_content} se muestran ejemplos de contenidos en HTML.

				\begin{figure}[htb]%

					\begin{lstlisting}						
<!-- Texto -->
<p>Un texto</p>

<!-- Imagen -->
<img src="imagen.jpg" alt="imagen">

<!-- Video -->
<video src="video.mp4"></video>

<!-- Audio -->
<audio src="sound.mp3"></audio>		 							 					
 					 			

					\end{lstlisting}
					\caption{Contenidos de un frame en HTML}
					\label{fig:frames_html_content}
				\end{figure}	  
		
		% subsection contenidos (end)


		\subsection{Overlays y Slide Items} % (fold)
		\label{sub:slide_items}
			Como se se vió en la definición \ref{def:slide_item}, un slide item es un contenido con un conjunto de overlays, para emplear la sintaxis de overlays vista en el ejemplo \ref{fig:overlay_set} se hace uso de los atributos \texttt{data} de HTML5, en este caso se definió el atributo \texttt{data-onslide}. Un slide item es un contenido de HTML con el atributo \texttt{data-onslide}, que tiene como valor el conjunto de overlays usando la sintaxis básica del ejemplo \ref{fig:overlay_set} o la sintaxis extendida vista en la sección \ref{sec:sintaxis_extendida}. Por ejemplo en la fig. se definen varios slide items usando la sintaxis básica.


				\begin{figure}[htb]%
					\begin{lstlisting}%

<img data-onslide="1-" src="test.jpg" alt="test">
<ul>
    <li data-onslide="3-">Desde el slide 3 hasta el final</li>
    <li data-onslide="4;6">En el slide 4 y 6</li>
    <li data-onslide="5;8">En el slide 5 y 8</li>
    <li data-onslide="3-5;7-8">Del 3 al 5 y del 7 al 8</li>
</ul> 

	
					\end{lstlisting}
					\caption{Forma de definir slide items con overlays en HTML5}
					\label{fig:slide_items_html}
				\end{figure}
			Un slide item está mostrado en todos los slides si no tiene overlays, o sea, si un elemento HTML no tiene el atributo \texttt{data-onslide} se verá en todas las transiciones del frame al que pertenece; de tenerlo, se ocultará en todos los slides a no ser que sus overlays determinen otra cosa.


			Para determinar los slides de un frame, \textit{beampress.js} primero analiza todos elementos HTML hijos de dicho frame y en dependencia de cada conjuntos de overlays, obtiene el menor y el mayor slide que son el primero y el último respectivamente. Para analizar cada overlay con la sintaxis básica, se utiliza la siguiente expresión regular: \texttt{/([1\textendash 9]\textbackslash d*)?(\textendash )?([1\textendash 9]\textbackslash d*)?/} para agrupar los límites inferior y superior, nótese que estos pueden no estar al igual que puede ser sólo un número sin \texttt{\textendash}. 

			Un overlay es un objeto JSON con el mismo formato de la fig. \ref{fig:json_format} y teniendo en cuenta la definición \ref{def:new_overlay} determina una función de transición siguiente y una función de transición anterior para uno o varios slides. Beampress transforma la sintaxs básica en la sintaxis extendida de la sección \ref{sec:sintaxis_extendida}. 

			Por ejemplo, si se tiene el siguiente conjunto de overlays: $$-2; 4$$ \textit{beampress.js} lo convierte en los overlays de la fig. \ref{fig:overlay_set_beampress}. Las funciones elementales vistas en el listado TAL se implementaron en \textit{beampress.js} de forma tal que modifican el atributo \texttt{opacity} del elemento en cuestión.

				\begin{figure}[htb]%
					\begin{lstlisting}%

// Slide 1
{"1":{"siguiente":{"func": "mostrar", "args": {}},
      "anterior": {"func": "ocultar", "args": {}}}} 
// Slide 2
{"2":{"siguiente":{"func": "identidad", "args": {}},
      "anterior": {"func": "identidad", "args": {}}}}  
// Slide 3
{"3":{"siguiente":{"func": "ocultar", "args": {}},
      "anterior": {"func": "mostrar", "args": {}}}}  
// Slide 4
{"4":{"siguiente":{"func": "mostrar", "args": {}},
      "anterior": {"func": "ocultar", "args": {}}}}                                       				
	
					\end{lstlisting}
					\caption{Representación en beampress.js del conjunto de slides: \(-2; 4\)}
					\label{fig:overlay_set_beampress}
				\end{figure}			
					
		
			
			Hay que aclarar que si se definen los overlays usando la sintaxis extendida, \textit{beampress.js} parsea explícitamente el overlay para convertirlo en objeto JSON. Por ejemplo en la fig \ref{fig:extended_syntax_html} se muestran slide items que tienen definidos sus overlays con la sintaxis extendida.

			\begin{figure}[htb]%
				\begin{lstlisting}%

<!-- Slide 4 -->
<span data-onslide='{
    "4": {
        "next": {
            "func": "slowShowItem",
            "args": {}
        },
        "prev": {
            "func": "slowHideItem",
            "args": {}
        }
    }}'>
	Leve transición en el slide 4
</span>

<!-- Slide 2 -->
<span data-onslide='{
    "2": {
        "next": {
            "func": "slowShowItem",
            "args": {}
        },
        "prev": {
            "func": "slowHideItem",
            "args": {}
        }
    }}'>
	Leve transición en el slide 2
</span>

			
				\end{lstlisting}
				\caption{Uso de la sintaxis extendida en slide items.} 
				\label{fig:extended_syntax_html}
			\end{figure}

			En la fig. \ref{fig:basic_and_extended_syntax_html} se exhibe un slide item que tiene especificado sus overlays usando la sintaxis extendida y la sintaxis básica.


			\begin{figure}[htb]%
				\begin{lstlisting}%


<span data-onslide='-2;{
    "4": {
        "next": {
            "func": "slowShowItem",
            "args": {}
        },
        "prev": {
            "func": "slowHideItem",
            "args": {}
        }
    }}'>
	Hasta el slide 2 y luego en el slide 4
</span>

			
				\end{lstlisting}
				\caption{Uso de ambas sintaxis extendida en un slide item.} 
				\label{fig:basic_and_extended_syntax_html}
			\end{figure}			


			Las funciones de transición \texttt{slowShowItem} y \texttt{slowHideItem} utilizadas en la fig. \ref{fig:extended_syntax_html} no son las elementales pero están definidas en \textit{beampress.js}, con el fin de cumplir los requerimientos de la sección \ref{sec:requirements} se implementaron otras funciones además de las elementales. Como invariante estas funciones reciben como argumento el selector DOM del elemento que modifican, a continuación se describe el efecto que produce cada una en el elemento que modifican:

			\begin{itemize}
				\item \texttt{addStyle(\$el, args)}: Agrega los estilos de CSS que recibe como argumentos
				\item \texttt{removeStyle(\$el, args)}: Quita todos los estilos CSS definidos
				\item \texttt{slowShowItem(\$el, args)}: Realiza un \texttt{animate} de JQuery incrementando la opacidad hasta el 100 PORCIENTO del elemento con el tiempo especificado en los argumentos
				\item \texttt{slowHideItem(\$el, args)}: Realiza un \texttt{animate} de JQuery disminuyendo la opacidad del elemento con el tiempo especificado en los argumentos
				\item \texttt{playAudio(\$el, args)}: Reproduce el audio en el tiempo y con el volumen especificados en los argumentos
				\item \texttt{stopAudio(\$el, args)}: Detiene el audio
				\item \texttt{playVideo(\$el, args)}: Reproduce el video en el tiempo y con el volumen especificados en los argumentos
				\item \texttt{stopVideo(\$el, args)}: Detiene el video
				\item \texttt{slowShowFrame(\$el, args)}: Similar a \texttt{slowShowItem} pero además establece el valor de \texttt{display} en \texttt{Blovk}
				\item \texttt{slowHideFrame(\$el, args)}: Similar a \texttt{slowHideItem} pero además establece el valor \texttt{display} en \texttt{None}				
			\end{itemize}
			En el capítulo \ref{ch:extensibilidad} se explicará como se pueden agregar otras funciones que se definan así como el formato que deben tener.

			En \textit{beampress.js} un \texttt{slide item} es un objeto de Javascript ~\cite{book:1047044} de tipo \texttt{SlideItem} con las siguientes propiedades:

			\begin{itemize}
			\label{it:slide_item}
			 	\item \texttt{\$el}: selector JQuery del elemento HTML que representa
			 	\item \texttt{overlays}: lista de todos los objetos JSON que definien sus overlays
			 	\item función \texttt{next}: recibe el número del slide actual como argumento del frame mostrado, ejectua la función \textit{siguiente} cuyo nombre está determinado por el overlay que tiene como llave el slide del argumento 
			 	\item función \texttt{prev}: recibe el número del slide actual como argumento del frame mostrado y ejectua la función \textit{anterior}			 	
			 \end{itemize} 
				
		% subsection slide_items (end)	

		\subsection{Frames} % (fold)
		 \label{sub:frames}
		 
		  
			Un \textit{frame} es un contenedor con la clase \texttt{frame}, el orden en que se muestra cada uno es el mismo con el que se define. En la fig. \ref{fig:frames_html} se muestran varios frames.


				\begin{figure}[htb]%
					\begin{lstlisting}%

<section class="frame">Primer Frame</section>
<section class="frame">Segundo Frame</section>
<!-- ... -->
<section class="frame">Enésimo Frame</section>
					\end{lstlisting}
					\caption{Forma de definir frames en HTML}
					\label{fig:frames_html}
				\end{figure}

			Los slide items de cada contenedor frame se definen dentro del mismo, o sea, son sus hijos. Cada frame es un objeto de Javascript de tipo \texttt{Frame} que hereda de \texttt{SlideItem} adquiriendo las propiedades vistas en el listado \ref{it:slide_item}. Para organizar la presentación \textit{beampress.js} analiza cada frame usando el selector \texttt{\$(".frame")} y por cada uno crea un objeto \texttt{Frame}. Los slide items hijos del frame se obtienen con el selector \texttt{\$("[data-onslide]")} y por cada uno se crea un objeto \texttt{SlideItem}. 

			Un objeto \texttt{Frame} tiene una lista de objetos \texttt{SlideItems} y sus funciones de transición o transiciones de entrada y salida por defecto se hacen alterando su atributo \texttt{display}, que en caso de estar mostrado es \texttt{Block} y en caso de estar oculto es \texttt{None}. Un frame hace su transición de salida cuando se visualizan todos sus slides y el siguiente (en caso de existir) realiza su transición de entrada. En la fig. \ref{fig:frame_html_slides} se muestra un frame con varios slide items.

			\begin{figure}[htb]%
				\begin{lstlisting}%

<section class="frame">
	<h1>Frame</h1>
	<span data-onslide="1-">Texto</span>
	<span data-onslide="2-">Texto3</span>
	<span data-onslide="3-">Texto4</span>
</section>
				\end{lstlisting}
				\caption{Un frame en HTML con varios slide items}
				\label{fig:frame_html_slides}
			\end{figure}

			Como se explicó anteriormente, el slide item \texttt{<h1>Frame</h1>} siempre estará visible en todos los slides del frame de la fig. \ref{fig:frame_html_slides} porque no tiene el atributo \texttt{data-onslide}, de hecho, no se crea ningún objeto \texttt{SlideItem} porque no está en los resultados del selector \texttt{\$("[data-onslide]")}.


			En la sección \ref{sec:extensibilidad} se argumentó que un frame podía tener funciones de transición, en este caso, transición de entrada y salida. Usando la sintaxis extendida y los atributos \texttt{data-onshow} y \texttt{data-onhide} se puede efectuar dicha funcionalidad. En la fig. \ref{fig:frames_transitions} se muestran dos frames con transiciones de entrada y de salida.


		\begin{figure}[htb]%
			\begin{lstlisting}%

<section data-onhide='{"func": "slowHideFrame", "args": {}}' class="frame"></section>
<section data-onshow='{"func": "slowShowFrame", "args": {}}' class="frame"></section>
			\end{lstlisting}
		\caption{Transiciones entre frames}
		\label{fig:frames_transitions}
		\end{figure}	

		El plugin \textit{beampress.js} realiza las transiciones de los frames de forma similar a la manera en que las realiza con los slide items, pero sin depender de un slide.
		% subsection frames (end)

		\subsection{Presentación} % (fold)
		\label{sub:presentacion}
			Todos los frames de la presentación tienen que estar contenidos dentro de un contenedor general, en la fig. \ref{fig:main_container} se define un contenedor con clase \texttt{main-container}.
		
			\begin{figure}[htb]%
				\begin{lstlisting}%

<div class="main-container">
	<!-- Los Frames van aquí -->
</div>
				\end{lstlisting}
			\caption{Contenedor principal}
			\label{fig:main_container}
			\end{figure}				


			Como \textit{beampress.js} está basado en JQuery, para usarlo es necesario incluir alguna biblioteca de JQuery; un ejemplo completo de su uso se puede ver en la Fig. \ref{fig:ex5}

			\begin{figure}[htb]%
				\begin{lstlisting}%

<!DOCTYPE html>
<html>
	<head>
    	<title>Ejemplo</title>
	</head>
	<body>
		<div id="main-container">
			<section class="frame">
				<!-- Contenido del frame -->
			</section>
			<section class="frame">
				<!-- Contenido del frame -->
			</section>						
		</div>
	   <script src="js/jquery-1.10.2.min.js"></script>
	   <script src="js/beampress.js"></script>
	   <script>
	       $(document).ready(function(){     
	           $("#main-container").beampress();
	        });
	   </script>		
	</body>
</html>			
				\end{lstlisting}
			\caption{
				Ejemplo de presentación usando \textit{beampress.js}. 
				\label{fig:ex5} }
			\end{figure}	

			El estilo de la presentación queda a manos del usuario, por lo que se requiere previos conocimientos de HTML, CSS y Javascript, lo cual constituye indudablemente una cierta desventaja. 
		% subsection presentación (end)
	% section beampress (end)

	\section{Beampressk} % (fold)
	\label{sec:beampressk_imp}
	
	% section beampressk (end)
		
% chapter detalles_de_implementacion (end)