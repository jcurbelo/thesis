%!TEX root = ../thesis.tex
\chapter{Detalles de implementación} % (fold)
\label{cha:detalles_de_implementacion}
	En este capítulo se explicará cómo fueron implementados \textit{Beampress} y \textit{Beampressk} tomando en cuenta los requerimientos vistos en la sección \ref{sec:requerimientos_basicos_del_sistema_propuesto} y siguiendo las concepciones definidas en el capítulo anterior. Se describe cómo se confecciona una presentación en HTML (especie de tutorial) y cómo el plugin de JQuery \textit{beampress.js} la interpreta. Se explica cómo se implementan las funciones de transición y las características de dicho plugin.

	Se analiza, de \textit{Beampressk}, la implementación del servidor web en Flask, cómo se definen la vista de control, de presentación, el modo en que se utilizó Websocket como tecnología para un canal de comunicación bidireccional entre estas, así como las funcionalidades de \textit{Beampressk} como API REST.

	Al definir la presentación en ámbito web, se aprovechan las bondades de HTML5 y CSS3 para conformar las diapositivas separando contenido de forma. A continuación se explica cómo estructurar una presentación y se detallan las características del plugin de JQuery \textit{beampress.js}. 

	\section{Beampress} % (fold)
	\label{sec:beampress_imp}
	

		Como se describió en la sección \ref{sec:beampress} se utlizó un  \textit{patrón ligero} como diseño para la implementación de \textit{beampress.js}. 

		En esta sección se explicará cómo crear contenidos, slide items, overlays y frames.

		Para crear una presentación siguiendo las definiciones del capítulo anterior, se hace uso de las etiquetas y atributos de HTML.


		\subsection{Contenidos} % (fold)
		\label{sub:contenidos}
			Los contenidos de la presentación son cualquier elemento de HTML que se especifique, por ejemplo un \textit{párrafo} (\texttt{<p>}), una \textit{imagen} (\texttt{<img>}), un \textit{video} (\texttt{<video>}), un \textit{audio} (\texttt{<audio>}) o un \textit{canvas} (\texttt{<canvas>}). En la fig. \ref{fig:frames_html_content} se muestran ejemplos de contenidos en HTML.

				\begin{figure}[htb]%

					\begin{lstlisting}						
<!-- Texto -->
<p>Un texto</p>

<!-- Imagen -->
<img src="imagen.jpg" alt="imagen">

<!-- Video -->
<video src="video.mp4"></video>

<!-- Audio -->
<audio src="sound.mp3"></audio>		 							 					
 					 			

					\end{lstlisting}
					\caption{Contenidos de un frame en HTML}
					\label{fig:frames_html_content}
				\end{figure}	  
		
		% subsection contenidos (end)


		\subsection{Overlays y Slide Items} % (fold)
		\label{sub:slide_items}
			Según la definición \ref{def:slide_item}, un slide item es un contenido con un conjunto de overlays. Para emplear la sintaxis de overlays vista en el ejemplo \ref{ex:overlay_set} se hace uso de los atributos \texttt{data} de HTML5, para lo cual se definió el atributo \texttt{data-onslide}. Un slide item es un contenido de HTML con el atributo \texttt{data-onslide}, que tiene como valor el conjunto de overlays usando la sintaxis básica del ejemplo \ref{ex:overlay_set} o la sintaxis extendida vista en la sección \ref{subsec:sintaxis_extendida}. Por ejemplo en la fig. \ref{fig:slide_items_html} se definen varios slide items usando la sintaxis básica.


				\begin{figure}[htb]%
					\begin{lstlisting}%

<img data-onslide="1-" src="test.jpg" alt="test">
<ul>
    <li data-onslide="3-">Desde el slide 3 hasta el final</li>
    <li data-onslide="4;6">En el slide 4 y 6</li>
    <li data-onslide="5;8">En el slide 5 y 8</li>
    <li data-onslide="3-5;7-8">Del 3 al 5 y del 7 al 8</li>
</ul> 

	
					\end{lstlisting}
					\caption{Forma de definir slide items con overlays en HTML5}
					\label{fig:slide_items_html}
				\end{figure}
			Un slide item está mostrado en todos los slides si no tiene overlays, o sea, si un elemento HTML no tiene el atributo \texttt{data-onslide} se mostrará en todas las transiciones del frame al que pertenece. En caso de tenerlo, se ocultará en todos los slides a no ser que sus overlays determinen otra conducta.


			Para determinar los slides de un frame, \textit{beampress.js} primero analiza todos elementos HTML hijos de dicho frame y en dependencia de cada conjuntos de overlays, obtiene el menor y el mayor slide, que son el primero y el último respectivamente.

			Para analizar cada overlay con la sintaxis básica, se utiliza la siguiente expresión regular: \texttt{/([1\textendash 9]\textbackslash d*)?(\textendash )?([1\textendash 9]\textbackslash d*)?/} para agrupar los límites inferior y superior. Nótese que estos pueden no estar al igual que puede ser sólo un número sin \texttt{\textendash}. 

			Un overlay es un objeto JSON con el mismo formato de la fig. \ref{fig:json_format} y teniendo en cuenta la definición \ref{def:new_overlay}, determina una función de transición siguiente y una función de transición anterior para uno o varios slides. El plugin \textit{beampress.js} transforma la sintaxs básica en la sintaxis extendida de la sección \ref{subsec:sintaxis_extendida}. 

			Por ejemplo, si se tiene el siguiente conjunto de overlays: $$-2; 4$$ \textit{beampress.js} lo convierte en los overlays de la fig. \ref{fig:overlay_set_beampress}. Las funciones elementales se implementaron en \textit{beampress.js} de forma tal que modifican el atributo \texttt{opacity} del elemento en cuestión.

				\begin{figure}[htb]%
					\begin{lstlisting}%

// Slide 1
{"1":{"siguiente":{"func": "mostrar", "args": {}},
      "anterior": {"func": "ocultar", "args": {}}}} 
// Slide 2
{"2":{"siguiente":{"func": "identidad", "args": {}},
      "anterior": {"func": "identidad", "args": {}}}}  
// Slide 3
{"3":{"siguiente":{"func": "ocultar", "args": {}},
      "anterior": {"func": "mostrar", "args": {}}}}  
// Slide 4
{"4":{"siguiente":{"func": "mostrar", "args": {}},
      "anterior": {"func": "ocultar", "args": {}}}}                                       				
	
					\end{lstlisting}
					\caption{Representación en beampress.js del conjunto de slides \texttt{-2; 4}}
					\label{fig:overlay_set_beampress}
				\end{figure}			
					
		
			
			Debe aclararse que al definir los overlays usando la sintaxis extendida, \textit{beampress.js} parsea explícitamente el overlay, para convertirlo en objeto JSON. Por ejemplo en la fig \ref{fig:extended_syntax_html} se muestran slide items que tienen definidos sus overlays con la sintaxis extendida.

			\begin{figure}[htb]%
				\begin{lstlisting}%

<!-- Slide 4 -->
<span data-onslide='{
    "4": {
        "next": {
            "func": "slowShowItem",
            "args": {}
        },
        "prev": {
            "func": "slowHideItem",
            "args": {}
        }
    }}'>
	Leve transición en el slide 4
</span>

<!-- Slide 2 -->
<span data-onslide='{
    "2": {
        "next": {
            "func": "slowShowItem",
            "args": {}
        },
        "prev": {
            "func": "slowHideItem",
            "args": {}
        }
    }}'>
	Leve transición en el slide 2
</span>

			
				\end{lstlisting}
				\caption{Uso de la sintaxis extendida en slide items.} 
				\label{fig:extended_syntax_html}
			\end{figure}

			En la fig. \ref{fig:basic_and_extended_syntax_html} se exhibe un slide item que tiene especificado sus overlays usando la sintaxis extendida y la sintaxis básica.


			\begin{figure}[htb]%
				\begin{lstlisting}%

<span data-onslide='-2;{
    "4": {
        "next": {
            "func": "slowShowItem",
            "args": {}
        },
        "prev": {
            "func": "slowHideItem",
            "args": {}
        }
    }}'>
	Hasta el slide 2 y luego en el slide 4
</span>

			
				\end{lstlisting}
				\caption{Uso de ambas sintaxis extendida en un slide item.} 
				\label{fig:basic_and_extended_syntax_html}
			\end{figure}			


			Las funciones de transición \texttt{slowShowItem} y \texttt{slowHideItem} utilizadas en la fig. \ref{fig:extended_syntax_html} no son las elementales, pero están definidas en \textit{beampress.js}. Con el fin de cumplir los requerimientos de la sección \ref{sec:requerimientos_basicos_del_sistema_propuesto} se implementaron otras funciones además de las elementales. Como invariante estas funciones reciben en sus argumentos el selector DOM del elemento que modifican. A continuación se describe el efecto que produce cada una en el elemento que modifican:

			\begin{itemize}
				\item \texttt{addStyle(\$el, args)}: Agrega los estilos de CSS que recibe como argumentos
				\item \texttt{removeStyle(\$el, args)}: Elimina todos los estilos CSS definidos
				\item \texttt{slowShowItem(\$el, args)}: Realiza un \texttt{animate} de JQuery incrementando la opacidad hasta el 100 PORCIENTO del elemento con el tiempo especificado en los argumentos
				\item \texttt{slowHideItem(\$el, args)}: Realiza un \texttt{animate} de JQuery disminuyendo la opacidad del elemento con el tiempo especificado en los argumentos
				\item \texttt{playAudio(\$el, args)}: Reproduce el audio en el tiempo y con el volumen especificados en los argumentos
				\item \texttt{stopAudio(\$el, args)}: Detiene el audio
				\item \texttt{playVideo(\$el, args)}: Reproduce el video en el tiempo y con el volumen especificados en los argumentos
				\item \texttt{stopVideo(\$el, args)}: Detiene el video
				\item \texttt{slowShowFrame(\$el, args)}: Similar a \texttt{slowShowItem} pero además establece el valor de \texttt{display} en \texttt{Blovk}
				\item \texttt{slowHideFrame(\$el, args)}: Similar a \texttt{slowHideItem} pero además establece el valor \texttt{display} en \texttt{none}				
			\end{itemize}
			En el capítulo \ref{cha:extensibilidad} se explicará la manera en que se pueden agregar otras funciones que se definan, así como el formato que deben tener.

			En \textit{beampress.js} un \texttt{slide item} es un objeto de Javascript ~\cite{book:1047044} de tipo \texttt{SlideItem} con las siguientes propiedades:

			\begin{itemize}
			\label{it:slide_item}
			 	\item \texttt{\$el}: selector JQuery del elemento HTML que representa
			 	\item \texttt{overlays}: lista de todos los objetos JSON que definien sus overlays
			 	\item función \texttt{next}: recibe el número del slide actual como argumento del frame mostrado. Ejectua la función \textit{siguiente}, cuyo nombre está determinado por el overlay que tiene como llave el slide del argumento 
			 	\item función \texttt{prev}: recibe el número del slide actual como argumento del frame mostrado y ejectua la función \textit{anterior}			 	
			 \end{itemize} 
				
		% subsection slide_items (end)	

		\subsection{Frames} % (fold)
		 \label{sub:frames}
		 
		  
			Un \textit{frame} es un contenedor con la clase \texttt{frame}, y el orden en que se muestra cada uno es el mismo con el que se define. En la fig. \ref{fig:frames_html} se muestran varios frames.


				\begin{figure}[htb]%
					\begin{lstlisting}%

<section class="frame">Primer Frame</section>
<section class="frame">Segundo Frame</section>
<!-- ... -->
<section class="frame">Enésimo Frame</section>
					\end{lstlisting}
					\caption{Forma de definir frames en HTML}
					\label{fig:frames_html}
				\end{figure}

			Los slide items de cada contenedor frame se definen dentro del mismo, o sea, son sus hijos. Cada frame es un objeto de Javascript de tipo \texttt{Frame} que hereda de \texttt{SlideItem}.
			 % para adquirir las propiedades vistas en el listado \ref{it:slide_item}. 

			Con el propósito de organizar la presentación \textit{beampress.js} analiza cada frame empleando el selector \texttt{\$(".frame")} y por cada uno crea un objeto \texttt{Frame}. Los slide items de un frame, o sea, todos los elementos HTML hijos de un contenedor frame, se obtienen con el selector \texttt{\$("[data-onslide]")} y por cada uno se crea un objeto \texttt{SlideItem}. Las instancias de \texttt{Frame} son almacenadas en una lista y en todo momento se sabe qué frame está mostrado y en qué slide se encuentra con los índices \texttt{currentFrame} y \texttt{currentSlide} respectivamente.

			Un objeto \texttt{Frame} tiene una lista de objetos \texttt{SlideItems}. Sus funciones de transición o transiciones de entrada y salida por defecto se hacen alterando su atributo \texttt{display}, que en caso de estar mostrado es \texttt{block} y en caso contrario es \texttt{none}. Un frame hace su transición de salida cuando se visualizan todos sus slides y el siguiente (en caso de existir) realiza su transición de entrada. En la fig. \ref{fig:frame_html_slides} se muestra un frame con varios slide items.

			\begin{figure}[htb]%
				\begin{lstlisting}%

<section class="frame">
	<h1>Frame</h1>
	<span data-onslide="1-">Texto</span>
	<span data-onslide="2-">Texto3</span>
	<span data-onslide="3-">Texto4</span>
</section>
				\end{lstlisting}
				\caption{Un frame en HTML con varios slide items}
				\label{fig:frame_html_slides}
			\end{figure}

			El slide item \texttt{<h1>Frame</h1>} de la fig. \ref{fig:frame_html_slides} siempre estará visible en todos los slides del frame porque no tiene definido un conjunto de overlays, o sea, no presenta el atributo \texttt{data-onslide}. No se crea ningún objeto \texttt{SlideItem} porque no está en los resultados del selector \texttt{\$("[data-onslide]")}.


			En la sección \ref{subsec:sintaxis_extendida} se argumentó que un frame podía tener funciones de transición, en este caso, transición de entrada y salida. Usando la sintaxis extendida y los atributos \texttt{data-onshow} y \texttt{data-onhide} se puede efectuar dicha funcionalidad. En la fig. \ref{fig:frames_transitions} se muestran dos frames con transiciones de entrada y de salida.


		\begin{figure}[htb]%
			\begin{lstlisting}%

<section data-onhide='{"func": "slowHideFrame", "args": {}}' class="frame"></section>
<section data-onshow='{"func": "slowShowFrame", "args": {}}' class="frame"></section>
			\end{lstlisting}
		\caption{Transiciones entre frames}
		\label{fig:frames_transitions}
		\end{figure}	

		El plugin \textit{beampress.js} realiza las transiciones de los frames de forma similar a la manera en que las realiza con los slide items, pero sin depender de un slide.
		% subsection frames (end)

		\subsection{Transiciones} % (fold)
		\label{sub:transiciones}
			Para realizar las transiciones o cambio de slides, se implementaron las funciones \texttt{next()} y \texttt{prev()} que efectúan las transiciones siguiente y anterior respectivamente. Estas funciones están registradas a distintos eventos, de forma tal que se ejecutan en dependecia de la interacción del usuario con el navegador o de la interacción de algún otro dispotivo utilizando Websocket. Esto último se explicará con más detalle en la sección \ref{sec:beampressk_imp}.
		% subsection transiciones (end)

		\subsection{Presentación} % (fold)
		\label{sub:presentacion}
			Todos los frames de la presentación tienen que estar contenidos dentro de un contenedor general. En la fig. \ref{fig:main_container} se define un contenedor con clase \texttt{main-container}.
		
			\begin{figure}[htb]%
				\begin{lstlisting}%

<div class="main-container">
	<!-- Los Frames van aquí -->
</div>
				\end{lstlisting}
			\caption{Contenedor principal}
			\label{fig:main_container}
			\end{figure}				


			Como \textit{beampress.js} está basado en JQuery, para usarlo es necesario incluir alguna biblioteca de JQuery. Un ejemplo completo de su uso se puede ver en la Fig. \ref{fig:ex5}

			\begin{figure}[htb]%
				\begin{lstlisting}%

<!DOCTYPE html>
<html>
	<head>
    	<title>Ejemplo</title>
	</head>
	<body>
		<div id="main-container">
			<section class="frame">
				<!-- Contenido del frame -->
			</section>
			<section class="frame">
				<!-- Contenido del frame -->
			</section>						
		</div>
	   <script src="js/jquery-1.10.2.min.js"></script>
	   <script src="js/beampress.js"></script>
	   <script>
	       $(document).ready(function(){     
	           $("#main-container").beampress();
	        });
	   </script>		
	</body>
</html>			
				\end{lstlisting}
			\caption{
				Ejemplo de presentación usando \textit{beampress.js}. 
				\label{fig:ex5} }
			\end{figure}	

			El estilo de la presentación queda sujeto al usuario, por lo que se requieren previos conocimientos de HTML, CSS y Javascript. Como extensión a la propuesta de este trabajo se implementó el estilo Warsaw de \textit{Beamer} con el CSS \texttt{warsaw.css} que se mencionará también en la siguiente sección.
		% subsection presentación (end)
	% section beampress (end)

	\section{Beampressk} % (fold)
	\label{sec:beampressk_imp}
		Se implementó una aplicación de \textit{Flask} para manejar todos los pedidos que el servidor web recibe usando el protocolo Web Server Gateway Interface (WSGI). La instancia de la aplicación es un objeto de clase \textit{Flask}.

		La aplicación asocia URLs con funciones de Python que tenga definidas. Dicha asociación se llama \textit{ruta} y en esta sección se asume \texttt{beampress.com} como URL base.
		
		\subsection{Estructura de la aplicación} % (fold)
		\label{sub:estructura_de_la_aplicacion}
			Todas las vistas que se definieron en la sección TAL se conforman usando el sistema de templates \textit{Jinja2}, en la fig. \ref{sub:estructura_de_la_aplicacion} se muestra la ubicación de cada template.
			Para interpretar las presentaciones, o sea, interpretar los frames, slides, slide items, overlays y las funciones de transición se usa el plugin \textit{beampress.js} cuyas características se argumentaron en la sección anterior. El uso de \textit{SocketIO} ~\cite{socketio} se realiza con la biblioteca \textit{socket.io..min.js}, y para eliminar los 300ms de espera en los navegadores móviles desde el momento en que se toca un botón para disparar el evento \texttt{click}, se usa \texttt{fastclick.js} ~\cite{fastclick}. La razón de este retraso es que el navegador está esperando para determinar si se hace un doble click ~\cite{fior}.
			\begin{figure}[htb]%
				\dirtree{%
				.1 beampressk.
				.2 static \ldots{} \begin{minipage}[t]{8cm}
									\textit{Todos los recursos se localizan aquí}{.}
								\end{minipage}.
				.3 css.
				.4 warsaw{.}css \ldots{} \begin{minipage}[t]{8cm}
									\textit{CSS para el estilo de Warsaw}{.}
								\end{minipage}.	
				.3 js.
				.4 beampress{.}js \ldots{} \begin{minipage}[t]{8cm}
									\textit{Plugin de Beampress}{.}
								\end{minipage}.
				.4 socket{.}io{.}min{.}js \ldots{} \begin{minipage}[t]{8cm}
									\textit{Biblioteca para SocketIO}{.}
								\end{minipage}.
				.4 fastclick{.}js \ldots{} \begin{minipage}[t]{8cm}
									\textit{Biblioteca Fastclick}{.}
								\end{minipage}.
				.2 templates.
				.3 nombre\_de\_pres \ldots{} \begin{minipage}[t]{8cm}
									\textit{Ejemplo de una presentación}{.}
								\end{minipage}.
				.4 index{.}html.
				.3 warsaw\_base{.}html \ldots{} \begin{minipage}[t]{8cm}
									\textit{Template base para todas las presentaciones con estilo Warsaw}{.}
								\end{minipage}.
				.3 index{.}html \ldots{} \begin{minipage}[t]{8cm}
									\textit{Template de la vista de control}{.}
								\end{minipage}.	
				.2 app{.}py \ldots{} \begin{minipage}[t]{8cm}
									\textit{La aplicación de Flask}{.}
								\end{minipage}.				
				}

			\caption{Estructura de Beampressk}
			\label{fig:beampressk_structure}
			\end{figure}		
		% subsection estructura_de_la_aplicación (end)

		\subsection{Presentaciones} % (fold)
		\label{sub:presentaciones}

			Por convenio un template de presentación debe llamarse \texttt{index.html} y estar dentro de una carpeta con el nombre de la presentación contenida en el directorio \texttt{templates} como se muestra en la fig. \ref{sub:estructura_de_la_aplicacion}. Para mostrar cada presentación se usa la siguiente URL: \\
			\texttt{beampressk.com/presentation/nombre\textendash de\textendash presentacion} y esta es intepretada por la aplicación de \textit{Flask} (\texttt{app.py}) con la función \texttt{show} de la fig.\ref{fig:presentation_routes}.

			Si el servidor tiene varios clientes y al mismo tiempo acceden a la misa URL de una presentación, esta se visualizará simultáneamente en varios dispositivos. 

			\begin{figure}[htb]%
				\begin{lstlisting}[language=Python]%

# Presentaciones
@app.route('/presentation/<name>')
def show(name):
    return render_template('%s/index.html' % name)
				\end{lstlisting}
			\caption{Rutas dinámicas para las presentaciones}
			\label{fig:presentation_routes}
			\end{figure}

		% subsection presentaciones (end)

		\subsection{Vista de control} % (fold)
		\label{sub:vista_de_control}
			El template de la vista de control se encuentra en \texttt{templates/index.html} como se muestra en la fig. \ref{sub:estructura_de_la_aplicacion} y la URL de la misma es la URL raíz: \texttt{beampress.com}. La función \texttt{index} de la fig. \ref{fig:control_view_code} es llamada luego de mapearse la URL raíz.

			\begin{figure}[htb]%
				\begin{lstlisting}[language=Python]%

# Vista de control
@app.route('/')
def index():
    return render_template('index.html')
				\end{lstlisting}
			\caption{Ruta para la vista de control}
			\label{fig:control_view_code}
			\end{figure}			
		
		% subsection vista_de_control (end)

		\subsection{Websocket} % (fold)
		\label{sub:websocket}

		En la sección TAL se definió que la comunicación entre la vista de control y la vista de presentación tenía que ser instantánea. Como se vió en la sección \ref{sec:beampressk} se utilizó websocket para establecer un canal de comunicación bidireccional entre estas vistas. Para ello se incluyó la extensión de Flask \textit{Flask-SocketIO} ~\cite{flasksocket} en la aplicación de Beampressk que establece una conexión con los clientes que tengan la biblioteca \textit{Socket.IO}.

			\subsubsection{Mensajes de comunicación} % (fold)
			\label{ssub:mensajes_de_comunicacion}
				Para establecer la comunicación entre la vista de control y la vista de presentaciones se usan los mensajes de \textit{SocketIO}. Estos mensajes son registrados con \textit{Flask-SocketIO} de forma similar a la manera en que registra las \textit{rutas} con funciones en Python. 

				De esta manera la aplicación de \textit{Flask} funciona como intermediaria entre la vista de control y la vista de presentación. Los mensajes definidos en la tabla \ref{tab:messages} se muestran en la fig. \ref{fig:control_view_msgs} y \ref{fig:presentation_view_msgs} para la parte cliente y en la fig. \ref{fig:server_msgs} para la parte servidor.


			\begin{figure}[htb]%
				\begin{lstlisting}[language=JavaScript]%

// Evento ante 'respuesta actualiza'
socket.on('response', function(msg) {
    // Actualizar información acerca 
    // de la presentación
}); 

// Mensaje 'siguiente'
socket.emit('next');

// Mensaje 'anterior'
socket.emit('prev');

// Mensaje 'volumen'
socket.emit('volume', {data: <valor-volumen>});

// Mensaje 'live feed'
socket.emit('live_feed');

// Mensaje 'message'
socket.emit('hide_msg');
  
				\end{lstlisting}
			\caption{Mensajes de comunicación en la vista de control}
			\label{fig:control_view_msgs}
			\end{figure}

			\begin{figure}[htb]%
				\begin{lstlisting}[language=JavaScript]%

// Mensaje 'siguiente'
socket.on('next_response', function() {
	// Transición siguiente
	next();
	// Mensaje 'respuesta actualiza'
	socket.emit('update_info', {data: <información>});
});


// Mensaje 'anterior'
socket.on('prev_response', function() {
	// Transición anterior
	prev();
	// Mensaje 'respuesta actualiza'
	socket.emit('update_info', {data: <información>});
});


//Mensaje 'volumen'
socket.on('volume_response', function(msg) {
	// Actualizar volumen
});

//Mensaje 'live feed'
socket.on('live_feed_response', function() {
    // Mostrar u ocultar el live feed
});

//Mensaje 'message'
socket.on('msg', function(msg){
	// Mostrar mensaje
});

//Mensaje 'message'
socket.on('hide_msg_response', function (){
	// Ocultar mensaje
}); 
  
				\end{lstlisting}
			\caption{Mensajes de comunicación en la vista de presentación}
			\label{fig:presentation_view_msgs}
			\end{figure}

			\begin{figure}[htb]%
				\begin{lstlisting}[language=Python]%

# Mensaje respuesta actualiza
@socketio.on('update_info', namespace='/beampressk')
def beampressk_update_info(message):
    data = message.get('data', {})
    emit('response', {'data': data}, broadcast=True)

# Mensaje siguiente
@socketio.on('next', namespace='/beampressk')
def beampressk_next():
    emit('next_response', broadcast=True)

# Mensaje anterior
@socketio.on('prev', namespace='/beampressk')
def beampressk_prev():
    emit('prev_response', broadcast=True)

# Mensaje volumen
@socketio.on('volume', namespace='/beampressk')
def beampressk_volume(message):
    emit('volume_response', {'data': message.get('data', {})}, broadcast=True)

# Mensaje live feed
@socketio.on('live_feed', namespace='/beampressk')
def beampressk_live_feed():
    emit('live_feed_response', broadcast=True)

# Mensaje message
@socketio.on('hide_msg', namespace='/beampressk')
def beampressk_hide_msg():
    emit('hide_msg_response', broadcast=True)
  
				\end{lstlisting}
			\caption{Mensajes de comunicación en el servidor}
			\label{fig:server_msgs}
			\end{figure}							

			% subsubsection mensajes (end)		

		\subsection{Live Feed} % (fold)
		\label{sub:live_feed}

			La captura multimedia HTML (audio y video) se realiza con la API \texttt{getUserMedia()} que funciona en combinación con los tags de HTML5 \texttt{<audio>} y \texttt{<video>}. El primer parámetro de \texttt{getUserMedia()} es para especificar el tipo de medio al que se quiere acceder. Por ejemplo, si se quiere solicitar la cámara web, el primer parámetro debe ser \texttt{video}. Para usar tanto el micrófono como la cámara, se utiliza \texttt{video, audio}.

			Como se observó en las fig. \ref{fig:presentation_view_msgs}, \ref{fig:control_view_msgs} y \ref{fig:server_msgs} se puede activar o desactivar un live feed mediante los mensajes de Websocket. A contiunación se explicará el uso de esta funcionalidad a través de otro servicio web, así como la incorporación de otros contenidos en la presentación utilizando las rutas de la aplicación en \textit{Flask}.
		
		% subsection live_feed (end)
		
		% subsection websockets (end)

		\subsection{API REST} % (fold)
		\label{sub:api_rest}
			En la sección TAL se explicaron las características de un servicio REST, cuya concepción está centrada en la noción de recursos. Para almacenar información referente a la presentación como el frame y el slide actual, así como el volumen los elementos multimedia de la misma, se utiliza una estructura de memoria. Esta estrategia solo funciona adecuadamente cuando el servidor web que ejecuta la aplicación de \textit{Flask} no es \textit{multi hilos}. Esto es factible ya que \textit{Beampressk} no un servidor web en producción, con una gran cantidad de accesos.

			Para efectuar un método POST y emitir un mensaje de Websocket se usa la función de la fig. \ref{fig:post_method}. De esta manera se puede activar el live feed o mostrar mensajes de texto que se envíen en el pedido.

			\begin{figure}[htb]%
				\begin{lstlisting}[language=Python]%

# pedidos RESTfuls

# POST
@app.route('/action', methods=['POST'])
def beampressk_emit_task():
    action = request.json['action']
    if action == 'live_feed':
        socketio.emit('live_feed_response', namespace='/beampressk')
    elif action == 'msg':
        data = request.json['data']
        socketio.emit('msg', {'data': data}, namespace='/beampressk')
    return jsonify({'action': action}), 200  
  
				\end{lstlisting}
			\caption{Método POST}
			\label{fig:post_method}
			\end{figure}			

			
			
		% subsection api_rest (end)	
	% section beampressk (end)

	En este capítulo fueron analizados, de \textit{Beampress} sus características, sus funcionalidades y el modo en que es capaz de interpretar las presentaciones. Por su parte de \textit{Beampressk} se explicó su implementación como servidor web en Flask, se definieron las vistas de control y de presentación destacando la instantaneidad lograda en la comunicación entre ambas al configurar un canal bidireccinal utilizando Websocket.

	Seguidamente se mostrará cómo pude definirse cualquier función de animación en Javascript, vinculándose a \textit{Beampress}, así como una forma genérica de agregar funcionalidades para la API REST que provee \textit{Beampressk}.
		
% chapter detalles_de_implementacion (end)