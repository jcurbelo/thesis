%!TEX root = ../thesis.tex
% Preliminares
\chapter{Preliminares} % (fold)
\label{cha:preliminares}

	En este capítulo se mencionan algunos de los principales sistemas de presentación enfatizando en sus ventajas y desventajas; se ofrecen los requerimientos que dieron origen a \textit{Beampress} y \textit{Beampressk}; y se presentan las tecnologías que fueron utilizadas para la implementación de estas propuestas.	

	\section{Sistemas de presentaciones} % (fold)
	\label{sec:sistemas_de_presentaciones}
		Para crear y controlar presentaciones, ya sea con propósitos científicos, educacionales, de negocio, culturales o de otro ámbito informativo, existen diversas herramientas. A continuación, se esbozan algunas de las principales características de los sistemas de presentaciones más conocidos, incluyendo las principales ventajas y desventajas de cada uno de ellos.	
		\subsection{Microsoft Office PowerPoint} % (fold)
		\label{sub:microsoft_office_power_point}

			Es un software desarrollado por \textit{Microsoft}  que permite realizar presentaciones simples o complejas. Las presentaciones pueden contener texto, imágenes, efectos de sonidos, videos, así como animaciones para dar una sensación más interactiva al público. Las diapositivas están compuestas por una plantilla donde el usuario puede escoger una imagen de fondo, una tipografía específica, y las informaciones que desee. Para un usuario estándar (que siempre use el mouse), es cómodo este sistema porque es uno de los de más amplio uso y de fácil comprensión. 

		% section microsoft_office_power_point (end)

		\subsection{Beamer} % (fold)
		\label{sub:beamer}

			Es una clase de \LaTeX{} para crear presentaciones especialmente útil en casos de propósitos científicos. \LaTeX{} es un lenguaje de marcado, con fuertes capacidades para mostrar notaciones matemáticas que permite la división entre contenido y visualización. Como inconveniente debe señalarse que resulta mucho más complicado insertar animaciones, videos y sonidos que en \textit{PowerPoint}.

		% section beamer (end)		

		\subsection{Prezi} % (fold)
		\label{sub:prezi}

			Es una aplicación innovadora que permite al usuario manejar su presentación de manera creativa, ya que consiste en un lienzo (canvas), infinito donde se puede concebir la presentación, lo cual implica una gran desventaja: hay que dedicarle tiempo al diseño de la misma. Esto no pasa con \textit{PowerPoint} ni con \textit{Beamer}, donde el diseño está basado en ``diapositivas''. Prezi es una herramienta más dinámica y original que \textit{Microsoft Office PowerPoint} y permite copiar y pegar archivos originados con dicho sistema, siendo posible usar online o descargarse para luego usarlo offline.  Una característica de considerable importancia que incrementa las ventajas de su uso, es que permite hacer zoom en los detalles o incluso modificarlos, sin que necesariamente haya que realizar otra diapositiva. El resultado final es una presentación que simula una pequeña película. Aunque es manejable sin necesidad de un orden preestablecido, tiene como desventajas fundamentales que obligatoriamente se necesita una cuenta, y que las funcionalidades gratuitas son limitadas, por ejemplo: la edición offline de la presentación es pagada.

			% Prezi ofrece cuatro modalidades de privacidad en dependencia de la licencia de la que se disponga (gratiuta, de pago, cuenta \textit{enjoy} o \textit{EDUenjoy})		
		% section prezi (end)


		\subsection{Sistemas basados en HMTL, CSS y Javascript} % (fold)
		\label{sub:otros_sistemas}
			A continuación se muestran otros sistemas de presentaciones basados en HTML, CSS y javascript, que son alternativas de \textit{Prezi}, y sirvieron de base para el presente trabajo, por lo cual se menciona sus nombres y se hace referencia brevemente a la característica fundamental de cada uno.
			\begin{itemize}
				\item \textit{Impress.js} ~\cite{impress}: Herramienta que hace uso de las transiciones y transformaciones de CSS3, está inspirada en Prezi
				\item \textit{jmpress.js} ~\cite{jmpress}: Es una reimplementación de impress.js usando JQuery
				\item \textit{Reveal.js} ~\cite{reveal}: Framework de presentaciones HTML
			\end{itemize}
		% section otros_sistemas (end)
	
	% section sistemas_de_presentaciones (end)

		Después de mencionar algunos de los principales sistemas de presentaciones existentes, conviene señalar que la propuesta de este trabajo surge por la necesidad de mejorar el método de presentación del Proyecto Delta ~\cite{delta}. A continuación resumiremos el funcionamiento básico de dicho espacio cultural vinculado a la ciencia, el humor y la tecnología.

	\section{Proyecto Delta} % (fold)
	\label{sec:proyecto_delta}

			El Proyecto Delta es un espacio cultural que vincula la ciencia, la tecnología y el humor. En cada espectáculo o puesta en escena su conductor principal escoge un tema específico, acerca del cual se proyectan diapositivas y materiales audiovisuales previamente seleccionados. Dicho evento cultural tiene la peculiaridad de que el público participa mediante un sistema de mensajes de texto, que pueden ser comentarios o respuestas a las interrogantes que se realizan en el transcurso del espectáculo. Además del presentador en escena, se visualiza y se escucha en la pantalla principal a otra persona que no se encuentra en el escenario. Dicha intervención en el espectáculo se realiza con la ayuda de una cámara y un micrófono y se le denomina \textit{live feed}.


			Desde hace varios años, \textit{Beamer} es el sistema de presentaciones con el que se confeccionan las diapositivas del Proyecto Delta. La práctica demostró que no era factible la inclusión de audios, videos y animaciones en la propia presentación sin la ayuda de un colaborador que controlara dichos elementos, lo que en ocasiones entorpecía el flujo del espectáculo. 

			% cual representaba un reto a solucionar.	


			Otra dificultad del Proyecto Delta es la distancia que existe entre la computadora que contiene la presentación y el presentador de la misma. Dicha distancia imposibilita el uso de \textit{presentadores inalámbricos}, los cuales están diseñados para funcionar en un radio máximo de 30m ~\cite{tikitiki}, que es menor que la distancia ya mencionada, de aproximadamente 50m.

			Otros impedimentos constatados son los siguientes: la dificultad del presentador para el registro, control y seguimiento de otros elementos propios del especáculo, como los mensajes de texto que envía el público; el volumen del sonido, cuyo dispositivo de control se encuentra lejos del escenario, en una cabina donde no llega el sonido de lo que se está proyectando en el espectáculo.

			Por último, en ocasiones es necesario reorganizar la presentación de las diapositivas según los requerimientos de cada puesta en escena, algunos de los cuales son imprevisibles dado el dinamismo inherente al Proyecto Delta. Esto se explica por la adecuación que algunos imprevistos obligan a llevar a cabo (ausencia de un invitado, retraso en el inicio de los equipos de audio y de video, problemas en la sala de exhibición, demandas del público, etc.) Con el objetivo de mantener el ritmo del espectáculo, es necesario la posibilidad de alterar el orden preconcebido de las diapositivas, de forma tal que pueda obviarse la vista de algunas de ellas o regresar a alguna anterior.			

			Realizar todas estas accciones con una presentación hecha con beamer o con cualquiera de los sistemas existentes puede resultar en extremo complicado. A continuación se muestran los requisitos fundamentales que debe reunir la propuesta de este trabajo para dar solución a las necesidades ya analizadas de dicho espacio cultural. 
	% section proyecto_delta (end)

	\section{Requerimientos básicos del sistema propuesto} % (fold)
	\label{sec:requerimientos_basicos_del_sistema_propuesto}
		En esta sección se mencionan las características que debe cumplir el sistema que se propone, de manera tal que resuelva las necesidades actuales del Proyecto Delta.
		\begin{itemize}
				\item Separación entre contenido y forma. 

					Las diapositivas deben crearse de forma independiente al diseño de la presentación, o sea, el estilo, la tipografía, así como otros elementos visuales del contenido, no deben estar incluidos en el mismo. 

					% Esto se hace necesario ya que la conformación de las presentaciones debe ser lo más similar a \textit{Beamer}, que ha sido la herramienta utilizada para generar las diapositivas del Proyecto Delta y además, propicia dicha separación gracias a \LaTeX{}.

				\item Disponibilidad de cuantas animaciones se desee y la inclusión de audios y videos en las diapositivas. 

					Por las características del Proyecto Delta, los videos son parte usual de las presentaciones, y disponer de animaciones haría mucho más amenas las diapositivas. Esta es una de las problemáticas principales: con \textit{Beamer} es complicado utilizar animaciones, audio y video.

				\item Que esté en ámbito web. 

					Se requiere de un formato para las presentaciones que sea compatible con cualquier sistema, que sea a la vez moderno y que pueda interactuar con otros servicios web presentes en el espectáculo como los mensajes de texto que envía el público.


				\item Posibilidad de crear diapositivas sin tener que usar el mouse. 

					Este requisito se refiere a la comodidad que representa no tener que recurrir a ningún aditamento fuera del teclado. Utilizando \textit{combinaciones de teclas} y \textit{snippets de códigos}, se ahorra tiempo y se evita la dispersión del pensamiento \cite{donjones}.
		\end{itemize}	
	% section requerimientos_basicos_del_sistema_propuesto (end)


	Después de haber desglosado los requerimientos que se necesitan, se muestran las soluciones que los satisfacen, así como las herramientas y tecnologías empleadas en dicho propósito. 

	\section{Beampress} % (fold)
	\label{sec:beampress}
		Para lograr que el sistema de presentaciones que se propone estructure las mismas de forma similar a \textit{Beamer}, y cumplimente los requisitos mencionados en la sección \ref{sec:requerimientos_basicos_del_sistema_propuesto}, se implementó el plugin de JQuery \textit{beampress.js} para aprovechar las bondades de HTML5, CSS3 y Javascript.

		\subsection{Plugin beampress.js} % (fold)
		\label{sub:beampress_js}
			 Para la implementación de \textit{beampress.js} se utlizó un  \textit{patrón ligero} como diseño de dicho plugin \cite{smashingmagazine} que cumple con las siguientes características que son producto del uso de las prácticas más comunes:

			\begin{itemize}
				\item Uso de un punto y coma \texttt{;} antes de llamar la función principal, para evitar conflictos con otros códigos de Javascript que tengan ausencia de dicho signo
				\item Un objeto \texttt{defaults}, que tiene las configuraciones iniciales del plugin
				\item Un \textit{constructor sencillo} para la creación y la asignación de la lógica que se desea emplear con el elemento HTML seleccionado
				\item Posibilidad de extender las configuraciones (extender \texttt{defaults})
				\item Un ligero \textit{wrapper} al constructor para evitar problemas como múltiples instancias
			\end{itemize}		
		% subsection beampress_js (end)


		Al estar en presencia de una herramienta que se encuentra en ámbito web, se aprovechan las posibilidades que esto ofrece, dando solución a las necesidades vistas en la sección \ref{sec:proyecto_delta}. Por ejemplo, la inclusión de audios y videos se puede realizar gracias a los tags \texttt{audio} y \texttt{video} de HTML5; las animaciones se pueden implementar puramente con Javascript, o se pueden crear con la función \texttt{animate} de JQuery ~\cite{animate}. También con los \texttt{transform} de CSS3 ~\cite{csstransform} o con alguna otra herramienta, como GreenSock Animation Platform (GSAP) ~\cite{gsap}, que se basa en Javascript y es más rápida que JQuery ~\cite{jquery}. 


		El nombre \textit{Beampress} surge por la unión de los nombres \textit{Beamer} e \textit{Impress}, sistemas de presentaciones que fueron detallados en la sección \ref{sec:sistemas_de_presentaciones}. Es de destacar que con \textit{Beampress} se solucionan algunos de los problemas señalados como: uso de animaciones y empleo de materiales audiovisuales en las diapositivas.	
	% section beampress (end)

	En la siguiente sección se ofrece una solución para otra de las problemáticas vistas en la sección \ref{sec:proyecto_delta}: la imposibilidad de usar un \textit{presentador inalámbrico} debido a la gran distancia entre el conductor y la computadora que contiene la presentación.
	
	\section{Beampressk} % (fold)
	\label{sec:beampressk}
		Con el propósito de vencer el obstáculo que representa la distancia entre presentador y diapositivas, surge \textit{Beampressk}: un servidor web que posibilita el control de la presentación a través de una red. Si se trata de una conexión \textit{wifi}, se garantiza el manejo de las diapositivas con un dispositivo móvil desde una distancia determinada, que puede ser mayor que la máxima en que son efectivos los \textit{presentadores inalámbricos} (30m).

		Para realizar dicho manejo, \textit{Beampressk} provee una vista de control, concebida para mostrarse en un dispositivo móvil y la vista de las diapositivas (la presentación como tal), que se muestra en un proyector o bien en la pantalla de una computadora. En el capítulo \ref{cha:detalles_de_implementacion} se detalla su estructura, así como la implementación de esta herramienta.

		\textit{Beampressk} es un servidor web implementado en \textit{Flask} ~\cite{flask}, que es un framework web de \textit{Python} que se categoriza como micro framework, o sea, tiene poca o ninguna dependencia de bibliotecas externas salvo en casos que se requiera de las mismas. Dicho framework está basado en \textit{Werkzeug} que es un conjunto de herramientas para aplicaciones que estén fundamentadas con Web Server Gateway Interface (WSGI) ~\cite{wsgi}, que es una especificación para una interfaz simple y universal entre servidores web y aplicaciones web o frameworks de \textit{Python}. \textit{Flask} también usa \textit{Jinja2} como motor de \textit{templates}, lo que permite estructurar los \textit{templates} de las vistas de control y presentación.


		El hecho de que sea una herramienta sencilla constituye una gran ventaja, sobre todo teniendo en cuenta el propósito para el cual fue concebida: intermediaria entre la vista de presentación y de control.

		Para controlar la presentación en tiempo real, o sea, que las acciones (modificación de volumen, cambio de diapositivas, etc.) que efectúe el presentador desde el dispositivo móvil tengan efecto instantáneo en la presentación, se utilizó \textit{websocket} ~\cite{websocket} como tecnología para enviar/recibir la información requerida a través de un canal de comunicación bidireccional entre la vista de diapostivas y la vista de control.

		Para lograr esto, se utilizó la extensión de \textit{Flask} \textit{Flask-SocketIO} ~\cite{flasksocket} que implementa el protocolo de paso de mensajes brindado por la biblioteca \textit{SocketIO} ~\cite{socketio} de Javascript.

	 	Teniendo en cuenta que \textit{Beampressk} es un servidor web se decidió concebirlo como API REST para que interactuara con otros servicios web, principalmente con los mensajes de texto que envía el público. A continuación se explican las funcionalidades de \textit{Beampressk} como \textbf{API REST}.
		
		% section definiciones_de_beampressk (end)

		\section{API REST} % (fold)
		\label{sec:api_rest}


			La \textbf{Transferencia de Estado Representacional} (Representational State Transfer) o \textbf{REST} es una arquitectura simple y \textit{sin estados} que generalmente funciona a través del protocolo HTTP. En los últimos años REST ha emergido como el diseño de la arquitectura estándar para servicios web (Servicios RESTful) y APIs web.

			Conjuntamente con el concepto de un servicio RESTful está la noción de \textit{recursos}. Los recursos son representados por URIs, un cliente envía un pedido (request) a estas URIs usando los métodos definidos por el protocolo HTTP, y posiblemente el estado de dicho recurso cambie como resultado de dicho pedido. En el caso de \textit{Beampressk} un recurso es la vista de presentación, la vista de control, los mensajes que envía el público del Proyecto Delta y el live feed entre otros que se definan.

			Los pedidos HTTP son diseñados para afectar un recurso en diversas formas, en la tabla \ref{tab:request} se muestran pedidos HTTP con las acciones que provocan en \textit{Beampressk}.

			\begin{table}[b]
				\caption{Ejemplos de tipos de pedidos HTTP con sus acciones}
				\label{tab:request}
				\centering
			
				\begin{tabular}{| l | p{4.25cm} | p{6.75cm} |}
				\hline
			
				\hline
				\textbf{Método} & & \\ 
				\textbf{HTTP} & \textbf{Acción} & \textbf{Ejemplos} \\
				\hline
					GET & Muestra una presentación &  http://beampressk.com/presentation/jce \\ 
				\hline

				\hline
					GET & Muestra el panel de control &  http://beampressk.com \\ 
				\hline

				\hline
					POST & Crea y muestra un mensaje con los datos del pedido &  http://beampressk.com/actions/msg \\ 
				\hline

				\hline
					DELETE & Elimina el frame 4 &  http://beampressk.com/frames/delete/4 \\ 
				\hline										
			
				\hline
				\end{tabular}
			\end{table}		

		El nombre \textit{Beampressk} es la mezcla de los nombres de \textit{Beampress} y \textit{Flask}. La idea de esta unión es similar a la motivó a \textit{Beampress}, ya que utiliza los títulos de las herramientas que dieron su origen.

		Esta herramienta resuleve las principales problemáticas ocasionadas por la distancia, garantizando que el presentador tenga control de la presentación aún estando lejos de la computadora que contiene la misma.


		En este capítulo se describieron las necesidades que dieron origen a \textit{Beampress} y a \textit{Beampressk}, sus requerimientos, las soluciones que ofrecen y las tecnologías que se emplearon. También se argumentaron las características de los sistemas de presentaciones más conocidos, incorporando las más significaticas a la propuesta de este trabajo, siendo esto parte de uno de los objetivos específicos visto en la sección \ref{sec:objetivos}. En el capítulo \ref{cha:fundamentos_conceptuales} se hablará cómo son concebidos \textit{Beampress} y \textit{Beampressk}, y los conceptos principales en los que se basan.
	% section beampressk (end)






% chapter preliminares (end)