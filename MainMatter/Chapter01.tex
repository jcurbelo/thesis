% Preliminares
\chapter{Preliminares} % (fold)
\label{cha:preliminares}

	En este acápite se muestran algunos de los principales sistemas de presentación tomando en cuenta las ventajas y desventajas de cada uno de ellos. Ofreceremos los requerimientos que dieron origen a \textit{Beampress} y \textit{Beampressk}, así como los detalles de las tecnologías que fueron utilizadas para la implementación de estas propuestas.

	Para crear y controlar presentaciones, ya sea con propósitos científicos, educacionales, de negocio, culturales o de otro ámbito informativo, existen diversas herramientas. A continuación, como ya anunciamos, esbozaremos las características principales de los sistemas de presentaciones más conocidos, enfatizando en las ventajas y desventajas que se tuvo en cuenta para la elaboración de este trabajo.	

	\section{Sistemas de presentaciones} % (fold)
	\label{sec:sistemas_de_presentaciones}
		\subsection{Microsoft Office Power Point} % (fold)
		\label{sub:microsoft_office_power_point}

			Es un software desarrollado por \textit{Microsoft}  que permite realizar presentaciones simples o complejas. Las presentaciones pueden contener texto, imágenes, efectos de sonidos, videos, así como animaciones para dar una sensación más interactiva al público. Las diapositivas están compuestas por una plantilla donde el usuario puede escoger una imagen de fondo, una tipografía específica, y las informaciones que desee. Para un usuario estándar (que siempre use el mouse), es cómodo este sistema porque es uno de los de más amplio uso y de fácil comprensión. 

		% section microsoft_office_power_point (end)

		\subsection{Beamer} % (fold)
		\label{sub:beamer}

			Es una clase de \LaTeX para crear presentaciones especialmente útil en casos de propósitos científicos. \LaTeX es un lenguaje Turing completo, con fuertes capacidades para mostrar notaciones matemáticas que permite la división entre contenido y visualización. Como inconveniente debe señalarse que resulta mucho más complicado insertar animaciones, videos y sonidos que en \textit{PowerPoint}.	

		% section beamer (end)		

		\subsection{Prezi} % (fold)
		\label{sub:prezi}

			Es una aplicación innovadora que permite al usuario manejar su presentación de manera creativa, ya que consiste en un lienzo (canvas), infinito donde se puede concebir la presentación, lo cual implica una gran desventaja: hay que dedicarle tiempo al diseño de la misma. Esto no pasa con \textit{PowerPoint} ni con \textit{Beamer}, donde el diseño está basado en ``diapositivas''. Es más dinámica y original que \textit{Microsoft Office PowerPoint} y permite copiar y pegar archivos originados con dicho sistema, siendo posible usar online o descargarse para luego usarlo offline.  Una característica de considerable importancia que incrementa las ventajas de su uso, es que permite hacer zoom en los detalles o incluso modificarlos, sin que necesariamente haya que realizar otra diapositiva. El resultado final es una presentación que simula una pequeña película. Aunque es manejable sin necesidad de un orden preestablecido, tiene como desventajas fundamentales que obligatoriamente se necesita una cuenta, y que las funcionalidades gratuitas son limitadas, por ejemplo: la edición offline de la presentación es pagada.

			Prezi ofrece cuatro modalidades de privacidad en dependencia de la licencia de la que se disponga (gratiuta, de pago, cuenta \textit{enjoy} o \textit{EDUenjoy})		
		% section prezi (end)


		\subsection{Otros sistemas} % (fold)
		\label{sub:otros_sistemas}
			A continuación se muestran otros sistemas de presentaciones basados en HTML, CSS y javascript, que son alternativas de \textit{Prezi}, y sirvieron de base para el presente trabajo, por lo cual se menciona sus nombres y se hace referencia brevemente a la característica fundamental de cada uno.
			\begin{itemize}
				\item \textit{Impress.js}: Herramienta que hace uso de las transiciones y transformaciones de CSS3, está inspirada en Prezi
				\item \textit{jmpress.js}: Es una reimplementación de impress.js usando JQuery
				\item \textit{Reveal.js}: Framework de presentaciones HTML
			\end{itemize}
		% section otros_sistemas (end)
	
	% section sistemas_de_presentaciones (end)

		Después de analizar los principales sistemas de presentaciones existentes, conviene señalar que la propuesta de este trabajo surge por la necesidad de mejorar el método de presentación del Proyecto Delta, para lo cual resumiremos el funcionamiento básico de dicho espacio cultural vinculado a la ciencia, el humor y la tecnología.

	\section{Proyecto Delta} % (fold)
	\label{sec:proyecto_delta}

			El Proyecto Delta es un espacio cultural que vincula la ciencia, la tecnología y el humor. En cada espectáculo o puesta en escena su conductor principal escoge un tema específico, acerca del cual se proyectan diapositivas y materiales audiovisuales previamente seleccionados. Dicho evento cultural tiene la peculiaridad de que el público participa mediante un sistema de mensajes de texto, que pueden ser comentarios o respuestas a las interrogantes que se realizan en el transcurso del espectáculo. Además del presentador en escena, se visualiza y se escucha en la pantalla principal al ayudante colateral que no se encuentra en el escenario, a esta característica se le denomina \textit{live feed}.


			Desde hace varios años, se utilizaba \textit{Beamer} en la confección de las diapositivas de este espacio. La práctica demostró que no era factible la inclusión de audios, videos y animaciones en la propia presentación sin la ayuda de un colaborador que controlara dichos elementos, lo cual representaba un reto a solucionar.	


			Otra dificultad del Proyecto Delta es la larga distancia entre la computadora que contiene la presentación y el ejecutor de la misma, o sea, el presentador. Dicha distancia imposibilita el uso de \textit{presentadores inalámbricos}, los cuales están diseñados para funcionar en un radio máximo de 30m, que es menor que la distancia ya mencionada, de aproximadamente 50m. Todo esto impide que el presentador pueda controlar las diapositivas con algún \textit{presentador inalámbrico}.

			Otros impedimentos constatados son los siguientes: la pejiguera del presentador para el registro, control y seguimiento de otros elementos propios del especáculo, como los mensajes de texto que envía el público respondiendo preguntas o comentando la evolución del evento; el volumen del audio, cuyo dispositivo de control se encuentra lejos del escenario, en una cabina donde no llega el sonido de lo que se está proyectando en el espectáculo. Esto cobra importancia al utilizarse materiales audiovisuales con distintos niveles de volumen.

			Por último, el conductor necesita reorganizar la presentación de las diapositivas según los requerimientos de cada puesta en escena, algunos de los cuales son imprevisibles dado el dinamismo inherente al Proyecto Delta. Esto se explica por la adecuación que algunos imprevistos obligan a llevar a cabo (ausencia de un invitado, retraso en el inicio de los equipos de audio y de video, problemas en la sala de exhibición, demandas del público, etc.). En haras de mantener el ritmo del espectáculo, se hace necesaria la posibilidad de alterar el orden preconcebido de las diapositivas, de forma tal que pueda obviarse la vista de algunas de ellas o regresar a alguna anterior.			
	% section proyecto_delta (end)


	\section{Requerimientos básicos del sistema propuesto} % (fold)
	\label{sec:requerimientos_basicos_del_sistema_propuesto}
		En esta sección se habla de las especifidades y características que debe tener el sistema que se propone, de manera tal que resuelva las necesidades vistas en la sección \ref{sec:proyecto_delta}.
		\begin{itemize}
				\item Separación entre contenido y forma. Esto significa que las diapositivas se definan independientes del diseño de la presentación. Esto es un requisito ya que la conformación de las presentaciones debe ser lo más similar a la clase \textit{Beamer} de \LaTeX que es la herramienta utilizada para generar las diapositivas del Proyecto Delta
				\item Disponibilidad de cuantas animaciones se desee así como la inclusión de audios y videos en las diapositivas. Como vimos en \ref{sec:proyecto_delta}, esto era una de las problemáticas principales, con \textit{Beamer}, es bastante complicado utilizar animaciones, audio y video.
				\item Que esté en ámbito web. Esto permite conformar las presentaciones con independencia de estilo, gracias a la separacion que ofrece HTML5 y CSS3 y bla bla bla
				\item Posibilidad de crear diapositivas cómodamente (sin tener que usar el mouse). Permitir escribir la estrucuta de las diapositivas de forma sencilla empleando HTML5 como lenguaje. Esto es cómodo porque podemos tener la utilidad de algún sistema de templates como Jinja que nos organice el codigo..... tambien podemos generar dicho codigo con algun generador automatico de presentaciones....
		\end{itemize}	
	% section requerimientos_basicos_del_sistema_propuesto (end)
	Después de haber desglosado los requerimientos que se necesita, se muestran las soluciones que los satisfacen así como las herramientas y tecnologías empleadas. 

	\section{Beampress} % (fold)
	\label{sec:beampress}
		Para lograr que nuestro sistema de presentaciones estructure las mismas de forma similar a \textit{Beamer} y cumplimente con lo requisitos mencionados en la sección \ref{sec:requerimientos_basicos_del_sistema_propuesto}, se implementó el plugin de JQuery \textit{beampress.js} aprovechando las bondades de HTML5, CSS3 y Javascript. Los patrones de diseño así como los detalles de su estructura serán explicados más detalladamente en el capítulo \ref{cha:detalles_de_implementacion}.

		Como estamos en presencia de una herramienta que está en ámbito web, se aprovecha de las posibilidades que esto ofrece, dando solución a las necesidades vistas en la sección \ref{sec:proyecto_delta}, por ejemplo, la inclusión de audios y videos que se realiza gracias a los tags \textbf{audio} y \textbf{video} de HTML5; las animaciones se pueden emplear puramente con Javascript, o basandose en JQuery, que permite el uso de la función ``animate'', con los ``transform'' de CSS3 o con alguna otra herramienta, como  GreenSock Animation Platform (GSAP), basada en Javascript y más rápida que JQuery. 

		El nombre \textit{Beampress} surge por la unión de los nombres \textit{Beamer} e \textit{Impress}, sistemas de presentaciones que fueron detallados en la sección \ref{sec:sistemas_de_presentaciones}.	
	% section beampress (end)


	Muela de outro


	Muela de intro
	
	\section{Beampressk} % (fold)
	\label{sec:beampressk}
		Por todo esto surge \textit{Beampressk}: una herramienta que posibilita el control de las diapositivas a través de una conexión wifi, lo cual garantiza el manejo de las mismas desde una distancia determinada, que puede ser mayor que la distancia máxima en que son efectivos los \textit{presentadores inalámbricos} (30m).

		\textit{Beampressk} es un servidor implementado en \textit{Flask}, que es un framework web de Python que se categoriza como micro-framework, o sea, originalmente tiene poca o ninguna dependencia a bibliotecas externas. La ligereza que lo tipifica constituye una gran ventaja, sobre todo teniendo en cuenta el propósito para el cual fue concebido: intermediario entre las vistas de presentaciones y la vista de control.

		Para controlar la presentación en tiempo real, se utilizó \textit{websocket} como tecnología para enviar/recibir la información requerida a través de un canal de comunicación bidireccional entre la presentación y el servidor. Más específicamente, \textit{Flask-SocketIO} porque implementa el protocolo de paso de mensajes brindado por la biblioteca \textit{SocketIO} de Javascript. De esta manera, se pueden controlar algunos parámetros del flujo de la presentación como el frame (diapositiva) actual, el intervalo actual, y el control de volumen.

		Al tener un servidor web implementado, naturalmente hace función de \textit{API Rest}, por lo cual responde a métodos de petición \textit{HTTP} previamente definidos. En el caso que nos ocupa, Proyecto Delta, se hace posible mostrar tanto mensajes de texto provenientes de otro servidor.	
	% section beampressk (end)





% chapter preliminares (end)