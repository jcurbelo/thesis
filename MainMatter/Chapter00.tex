% Introduction

\begin{introduction}

	Las presentaciones y diapositivas, han pasado a formar parte de la vida cotidiana. Actualmente se usan no solo en la docencia o eventos científicos, sino en casi todos los escenarios de la vida, llegando incluso a espectáculos artísticos (Esto último se puede apreciar algunos esp ectáculos de Jeff Dunham, y todos los espectáculos del Proyecto Delta).


	Una forma en la que se puede medir la importancia y el impacto de las presentaciones en el mundo moderno, es la cantidad de libros que se publican sobre el tema, y aunque en la mayoría de estos libros, aclaran que las diapositivas no son la presentación, también aclaran que las diapositivas son un componente muy importante.


	Precisamente, por la importancia de las mismas, existen muchos sistemas que permiten preparar diapositivas, como Powerpoint, la clase Beamer de \LaTeX, o Prezi por citar a los más conocidos, aunque en los últimos años, han aparecido un bueno número de sistemas de diapositivas basados en HTML5 y CSS3, como Impress.js, Jmpress.js, Reveal.js y deck.js.


	Todos estos sistemas tienen ventajas y desventajas, que serán descritos en detalle en la sección Capítulo antecedentes, pero sirven de base y antecedentes al presente trabajo, en el que se pretende diseñar e implementar un sistema de diapositivas con énfasis en temas científicos (en particular Matemática y Computación), y que reúna la mayor cantidad de ventajas y comodidades de los existentes.

	\section*{Objetivos}

		El objetivo general de este trabajo es diseñar una aplicación web que permita mostrar y controlar diapositivas creadas con HTML5, CSS3 y Javascript y que simule todas las funcionalidades de Beamer y otros sistemas de presentaciones existentes. Además, debe tener varias vistas: la presentación que se muestra al público, y los paneles de control que permitan controlarla, y para lograrlo se han trazado los siguientes objetivos específicos:

		\begin{enumerate}
			\item Estudio de los sistemas de presentaciones existentes, para determinar sus ventajas y desventajas para la confección de diapositivas amenas con temáticas científicas, y qué características de los mismas incorporar al nuevo sistema
			\item Estudio de las funcionalidades y bondades de HTML5, Javascript y CSS3 para desarrollar un sistema de diapositivas
			\item Implementar el sistema de diapositivas con las características seleccionadas
			\item Implementar una aplicación web que permita controlar las diapositivas confeccionadas con esta herramienta
		\end{enumerate}

	\section*{Estructura del documento}
		Los resultados de este trabajo, y las aplicaciones desarrolladas se presentan en este documento que tiene la siguiente estructura. En el capítulo antecedentes se presentan las herramientas ya existentes para realizar presentaciones, sus principales características, las tecnologías que vamos a utilizar y por qué. El Capítulo 2 describe las utilidades y las funcionalidades que se desea que tenga nuestra aplicación. El Capítulo 3 es casi un tutorial, se explica cómo puede el usuario lograr cada una de las cosas que se presentaron en el capítulo 2. En el Capítulo 4 se muestran los detalles de implementación, orientadas a que un lector interesado, sepa cómo puede extender la aplicación, ya sea con nuevos efectos, o con nuevas funcionalidades. Finalmente, se presentan las conclusiones, recomendaciones y bibliografía consultada.



%---------------------------------------------------------------------------------



\end{introduction}




