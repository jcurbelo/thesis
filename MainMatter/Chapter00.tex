%!TEX root = ../thesis.tex
% Introduction
\begin{introduction}

	Las presentaciones y diapositivas han pasado a formar parte de la vida cotidiana. Actualmente se usan no solo en la docencia o eventos científicos, sino en casi todos los escenarios de la vida, llegando incluso a espectáculos artísticos, como se puede apreciar en algunas funciones de Jeff Dunham ~\cite{dunham}, y en el Proyecto Delta ~\cite{delta}.


	Una forma que permite medir la importancia y el impacto de las presentaciones en el mundo moderno, es la gran cantidad de referencias bibliográficas que se publican sobre el tema ~\cite{alley, duarte, tufte}. Es de señalar que aunque en la mayoría de estos textos se aclara que las diapositivas no son únicamente la presentación, también se puntualiza que estas consituyen un componente importante.

	Existen varios sistemas que permiten crear y controlar diapositivas, ejemplos de algunos de estos son: \textit{Microsoft Office PowerPoint} ~\cite{powerpoint}, \textit{Beamer}, que es la opción de \LaTeX{} para hacer diapositivas, \textit{Impress de LibreOffice} ~\cite{libreoffice}, Keynote para Mac ~\cite{keynote}, entre otros. También existen varios basados en HTML, CSS y Javascript, como son: \textit{Prezi} ~\cite{prezi}, Impress.js ~\cite{impress}, Reveal.js ~\cite{reveal}, jmpress.js ~\cite{jmpress}.

	El propósito de este trabajo es diseñar una aplicación web que permita mostrar y controlar diapositivas creadas con HTML5, CSS3 y Javascript y que simule algunas de las funcionalidades de Beamer ~\cite{beamer} y otros sistemas de presentaciones existentes. Dadas las características del proyecto cultural Delta (interacción con el público, proyección de videos, audios y animaciones), y teniendo en cuenta las dificultades que ofrecía el modelo actual, en términos de distancia entre el presentador y el panel de control, entre otras, surgió la necesidad de crear un sistema ágil y facilitador de presentaciones que permitiera el dinamismo que esta actividad científico recreativa exige.

	Como se verá en el capítulo \ref{cha:preliminares} algunos de los sistemas de diapositivas ya mencionados, sirven de base al presente trabajo y poseen ventajas de las cuales se ha apropiado, así como algunas dificultades que se ha evitado. Se adaptó la concepción de PowerPoint ~\cite{powerpoint} de incluir audios, videos y animaciones; se reprodujo el hecho de que Prezi ~\cite{prezi}, Impress.js ~\cite{impress} e jmpress.js ~\cite{jmpress} estén en ámbito web para aprovechar la separación entre contenido y forma que permite HTML. De Beamer se utilizó su manera de estructurar las presentaciones y al ser este último una clase de \LaTeX{} \cite{latex}, se incluyó su utilidad especial para propósitos científicos; más específicamente la opción de usar expresiones matemáticas. Si estamos en presencia de un sistema en ámbito web, esto se hace posible gracias a las diversas herramientas que permiten incluir \LaTeX, \textit{MathML}, o \textit{AsciiMath} en las páginas web \cite{mathjax, katex, phpmath}. 


	\section{Objetivos} % (fold)
	\label{sec:objetivos}

		El objetivo general de este trabajo es diseñar una aplicación web que permita mostrar y controlar diapositivas creadas con HTML5, CSS3 y Javascript. Para ello se han trazado los siguientes objetivos específicos:

		\begin{enumerate}
			\item Estudio de los sistemas de presentaciones existentes, para determinar sus ventajas y desventajas para la confección de diapositivas amenas con temáticas científicas, y qué características de los mismas incorporar al nuevo sistema
			\item Estudio de las funcionalidades y bondades de HTML5, Javascript y CSS3 para desarrollar un sistema de diapositivas
			\item Implementar el sistema de diapositivas llamado \textit{Beampress}, con las características seleccionadas
			\item Implementar una aplicación web en \textit{Flask} ~\cite{flask} que permita controlar las diapositivas confeccionadas con esta herramienta, denominado \textit{Beampressk}
		\end{enumerate}
	% section objetivos (end)

	\section*{Estructura del documento} % (fold)
	\label{sec:estructura_del_documento}
	
		Los resultados de este trabajo, y las aplicaciones desarrolladas se presentan en este documento que tiene la siguiente estructura. En el capítulo \ref{cha:preliminares} se presentan las herramientas ya existentes para realizar presentaciones, sus principales características, las tecnologías que vamos a utilizar y por qué. El Capítulo 2 describe las utilidades y las funcionalidades que se desea que tenga nuestra aplicación. El Capítulo 3 es casi un tutorial, se explica cómo puede el usuario lograr cada una de las cosas que se presentaron en el capítulo 2. En el Capítulo 4 se muestran los detalles de implementación, orientadas a que un lector interesado, sepa cómo puede extender la aplicación, ya sea con nuevos efectos, o con nuevas funcionalidades. Finalmente, se presentan las conclusiones, recomendaciones y bibliografía consultada.

	% section estructura_del_documento (end)		

%---------------------------------------------------------------------------------



\end{introduction}




