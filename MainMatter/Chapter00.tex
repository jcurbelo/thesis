% Introduction

\begin{introduction}

	Las presentaciones y diapositivas han pasado a formar parte de la vida cotidiana. Actualmente se usan no solo en la docencia o eventos científicos, sino en casi todos los escenarios de la vida, llegando incluso a espectáculos artísticos (Esto último se puede apreciar en algunos espectáculos de Jeff Dunham, y en el Proyecto Delta).


	Una forma que permite medir la importancia y el impacto de las presentaciones en el mundo moderno, es la gran cantidad de referencias bibliográficas que se publican sobre el tema. Es de señalar que aunque en la mayoría de estos textos se aclara que las diapositivas no son únicamente la presentación, también se puntualiza que estas consituyen un componente muy importante.

	Se propone diseñar una aplicación web que permita mostrar y controlar diapositivas creadas con HTML5, CSS3 y Javascript y que simule todas las funcionalidades de Beamer y otros sistemas de presentaciones existentes. Dadas las características del proyecto cultural Delta (interacción con el público, proyección de videos, audios y animaciones), y teniendo en cuenta las dificultades que ofrecía el modelo actual en términos de distancia entre el presentador y el panel de control, entre otras, surgió la necesidad de crear un sistema ágil y facilitador de presentaciones que permitiera el dinamismo que esta actividad científico-recreativa exige.

	Como se verá en el capítulo \ref{cha:preliminares} todos los sistemas de diapositivas sirven de base al presente trabajo y poseen las ventajas de las cuales nos hemos apropiado así como algunas dificultades que hemos evitado. Se adaptó la concepción de PowerPoint de incluir audios, videos y animaciones; se reprodujo el hecho de que Prezi esté en ámbito web aunque descartamos de este sistema su limitación al tener que preestablecer un diseño bien definido para la presentación. De la clase Beamer de \LaTeX  aprovechamos su utilidad especial para propósitos científicos y la manera de estructuar las presentaciones separando contenido de visualización.


	\section*{Objetivos}

		El objetivo general de este trabajo es diseñar una aplicación web que permita mostrar y controlar diapositivas creadas con HTML5, CSS3 y Javascript y que simule todas las funcionalidades de Beamer y otros sistemas de presentaciones existentes. Además, debe tener varias vistas: la presentación que se muestra al público, y los paneles de control que permitan controlarla, y para lograrlo se han trazado los siguientes objetivos específicos:

		\begin{enumerate}
			\item Estudio de los sistemas de presentaciones existentes, para determinar sus ventajas y desventajas para la confección de diapositivas amenas con temáticas científicas, y qué características de los mismas incorporar al nuevo sistema
			\item Estudio de las funcionalidades y bondades de HTML5, Javascript y CSS3 para desarrollar un sistema de diapositivas
			\item Implementar el sistema de diapositivas que llamaremos \textit{Beampress}, con las características seleccionadas
			\item Implementar una aplicación web en \textit{Flask} que permita controlar las diapositivas confeccionadas con esta herramienta, que llamaremos \textit{Beampressk}
		\end{enumerate}

	\section*{Estructura del documento}
	
		Los resultados de este trabajo, y las aplicaciones desarrolladas se presentan en este documento que tiene la siguiente estructura. En el capítulo \ref{cha:preliminares} se presentan las herramientas ya existentes para realizar presentaciones, sus principales características, las tecnologías que vamos a utilizar y por qué. El Capítulo 2 describe las utilidades y las funcionalidades que se desea que tenga nuestra aplicación. El Capítulo 3 es casi un tutorial, se explica cómo puede el usuario lograr cada una de las cosas que se presentaron en el capítulo 2. En el Capítulo 4 se muestran los detalles de implementación, orientadas a que un lector interesado, sepa cómo puede extender la aplicación, ya sea con nuevos efectos, o con nuevas funcionalidades. Finalmente, se presentan las conclusiones, recomendaciones y bibliografía consultada.



%---------------------------------------------------------------------------------



\end{introduction}




