%!TEX root = ../thesis.tex
% Introduction
\begin{introduction}

	Actualmente las presentaciones y diapositivas forman parte de la vida cotidiana. Se usan no solo en la docencia o eventos científicos, sino en casi todos los escenarios de la vida, llegando incluso a espectáculos artísticos, como se puede apreciar en algunas funciones de Jeff Dunham ~\cite{dunham}, y en el Proyecto Delta ~\cite{delta}.


	Una forma que permite medir la importancia y el impacto de las presentaciones en el mundo moderno, es la gran cantidad de referencias bibliográficas que se publican sobre el tema ~\cite{alley, duarte, tufte}. Es de señalar que aunque en la mayoría de estos textos se aclara que las diapositivas no son únicamente la presentación, también se puntualiza que estas constituyen un componente importante.


	Existen varios sistemas que permiten crear y controlar diapositivas, algunos ejemplos son: \textit{Microsoft Office PowerPoint} ~\cite{powerpoint}, \textit{Beamer}, que es una clase de \LaTeX{} para hacer diapositivas, \textit{Impress de LibreOffice} ~\cite{libreoffice}, \textit{Keynote} para Mac ~\cite{keynote}, entre otros. También existen varios en ámbito web como son: \textit{Prezi} ~\cite{prezi}, \textit{Impress.js} ~\cite{impress}, \textit{Reveal.js} ~\cite{reveal}, \textit{jmpress.js} ~\cite{jmpress}.

	El propósito de este trabajo es diseñar una aplicación web que permita mostrar y controlar diapositivas creadas con HTML5, CSS3 y JavaScript y que simule algunas de las funcionalidades de Beamer ~\cite{beamer} y otros sistemas de presentaciones existentes.

	Dadas las características del proyecto cultural Delta, que incluyen interacción con el público, proyección de videos, audios y animaciones, y teniendo en cuenta la dificultad principal que ofrece el modelo actual, que es la distancia entre el presentador y el panel de control, surgió la necesidad de crear un sistema ágil y facilitador de presentaciones que permitiera el dinamismo que esta actividad científico recreativa exige.

	Como se verá en el capítulo \ref{cha:preliminares}, algunos de los sistemas de diapositivas ya mencionados sirven de base al presente trabajo, y poseen tanto ventajas como desventajas. La propuesta que se propone en este trabajo adapta la concepción de PowerPoint ~\cite{powerpoint} de incluir audios, videos y animaciones; reproduce el hecho de que Prezi ~\cite{prezi}, Impress.js ~\cite{impress} e jmpress.js ~\cite{jmpress} estén en ámbito web para aprovechar la separación entre contenido y forma que permite HTML. De Beamer se utiliza su manera de estructurar las presentaciones y al ser este último una clase de \LaTeX{} \cite{latex}, se incluye su utilidad especial para propósitos científicos; más específicamente la opción de usar expresiones matemáticas. Si se está en presencia de un sistema en ámbito web, esto se hace posible gracias a las diversas herramientas que permiten incluir \LaTeX, \textit{MathML}, o \textit{AsciiMath} en las páginas web \cite{mathjax, katex, phpmath}. 


	\section*{Objetivos} % (fold)
	\label{sec:objetivos}

		El objetivo general de este trabajo es diseñar una aplicación web que permita mostrar y controlar diapositivas. Para ello se han trazado los siguientes objetivos específicos:

		\begin{enumerate}
			\item Estudiar los sistemas de presentaciones existentes, para determinar sus ventajas y desventajas para la confección de diapositivas amenas con temáticas científicas, y qué características de los mismas incorporar al nuevo sistema
			\item Estudiar las funcionalidades y bondades de HTML5, JavaScript y CSS3 para desarrollar un sistema de diapositivas
			\item Implementar el sistema de diapositivas llamado \textit{Beampress}, con las características seleccionadas
			\item Implementar una aplicación web en \textit{Flask} ~\cite{flask, flaskbook} que permita controlar las diapositivas confeccionadas con esta herramienta, denominado \textit{Beampressk}
		\end{enumerate}
	% section objetivos (end)

	% \section*{Estructura del documento} % (fold)
	% \label{sec:estructura_del_documento}
		El presente trabajo consta de cuatro capítulos donde se explican los antecedentes, los fundamentos conceptuales de este proyecto y sus detalles de implementación, se arriban a conclusiones y se sugieren recomendaciones. El capítulo \ref{cha:preliminares} se dedica a las características de las herramientas existentes para realizar presentaciones, y se explican las tecnologías utilizadas para la propuesta de este trabajo, así como las razones por las cuales fueron utilizadas según los requerimientos planteados en este proyecto. Los fundamentos conceptuales se describen en el capítulo \ref{cha:fundamentos_conceptuales}, mientras que en el capítulo \ref{cha:detalles_de_implementacion} se explican los detalles de implementación de \textit{Beampress} y \textit{Beampressk} derivados de las definiciones del capítulo anterior. Esto posibilita conformar una presentación que puede ser utillizada para diversos fines. El capítulo \ref{cha:extensibilidad} muestra cómo se permite la extensibilidad de las aplicaciones previamente explicadas, de forma que puedan obtenerse nuevas funcionalidades. Finalmente, se presentan las conclusiones, las recomendaciones y la bibliografía consultada.

	% section estructura_del_documento (end)		

%---------------------------------------------------------------------------------



\end{introduction}




