%%%{{{ Comments and the like

\usepackage{acronym}
\usepackage{bbm}
%%\usepackage{MnSymbol}
%%\usepackage{turnstile}
\usepackage[slide,ruled,algochapter,lined,vlined,linesnumbered]{algorithm2e}
\usepackage{tikz}
\usetikzlibrary{arrows,positioning,shapes.misc,shapes.symbols}
%%\usepackage{colortbl}
%%\usepackage[makeroom]{cancel}
\usepackage{ctable}

\newcommand{\R}{\mathbbm R}
\newcommand{\Rm}{\mathbbm{R}^{m}}
\newcommand{\Rn}{\mathbbm{R}^{n}}
\newcommand{\code}[1]{\texttt{#1}}
\newcommand{\codet}[1]{\texttt{\textbf{#1}}}
\newcommand{\htbl}[1]{\textsf{#1}}
\newcommand{\rec}[1]{ \tikz \node[rectangle,draw=red!25,thick,fill=red!20,align=center]{#1}; }
\newcommand{\fig}[2]{\includegraphics[width=#1\linewidth]{gfx//#2}} %width=\linewidth
\newcommand{\xcancel}[1]{\tikz[baseline=(X.base),opacity=.7]\node[cross out,draw=red,line width=.3ex](X){#1};}

\title{Beampress: una aplicación web para mostrar y controlar diapositivas }
\author{Jorge Roberto Curbelo Fernández}
\advisor{M.Sc. Fernando Raúl Rodríguez Flores}
\degree{Licenciado en Ciencia de la Computación}
\faculty{Facultad de Matemática y Computación}
\date{Junio de 2015}
\logo{gfx/uhlogo.pdf}
\makenomenclature

\hyphenation{he-rra-mien-tas Py-thon dot-Value}

%%\flushleft
%%\newenvironment{vardesc}[1]{%
%%\settowidth{\parindent}{#1:\ }
%%\makebox[0pt][r]{#1:\ }}{}
%%\begin{displaymath}
%%a^2+b^2=c^2
%%\end{displaymath}
%%\begin{vardesc}{Where}$a$,
%%$b$ -- are adjunct to the right
%%angle of a right-angled triangle.
%%$c$ -- is the hypotenuse of
%%the triangle and feels lonely.
%%$d$ -- finally does not show up
%%here at all. Isn't that puzzling?
%%\end{vardesc}

%%%}}}


%%% Local Variables: 
%%% mode: latex
%%% TeX-master: "thesis"
%%% End: 
