%%%{{{ Comments and the like

\usepackage{acronym}
\usepackage{bbm}
%%\usepackage{MnSymbol}
%%\usepackage{turnstile}
\usepackage[slide,ruled,algochapter,lined,vlined,linesnumbered]{algorithm2e}
\usepackage{tikz}
\usetikzlibrary{arrows,positioning,shapes.misc,shapes.symbols}
%%\usepackage{colortbl}
%%\usepackage[makeroom]{cancel}
\usepackage{ctable}

\usepackage{color}
\usepackage{listings}
\definecolor{lightgray}{rgb}{0.95, 0.95, 0.95}
\definecolor{darkgray}{rgb}{0.4, 0.4, 0.4}
\definecolor{purple}{rgb}{0.65, 0.12, 0.82}
\definecolor{editorGray}{rgb}{0.95, 0.95, 0.95}
\definecolor{editorOcher}{rgb}{0.5, 0.5, 1} % #FF7F00 -> rgb(239, 169, 0)
\definecolor{editorGreen}{rgb}{0, 0.5, 0} % #007C00 -> rgb(0, 124, 0)

%% Defining Languages
% CSS
\lstdefinelanguage{CSS}{
  keywords={color,background-image:,margin,padding,font,weight,display,position,top,left,right,bottom,list,style,border,size,white,space,min,width, transition:, transform:, transition-property, transition-duration, transition-timing-function},	
  sensitive=true,
  morecomment=[l]{//},
  morecomment=[s]{/*}{*/},
  morestring=[b]',
  morestring=[b]",
  alsoletter={:},
  alsodigit={-}
}

% JavaScript
\lstdefinelanguage{JavaScript}{
  morekeywords={typeof, new, true, false, catch, function, return, null, catch, switch, var, if, in, while, do, else, case, break},
  morecomment=[s]{/*}{*/},
  morecomment=[l]//,
  morestring=[b]",
  morestring=[b]'
}

% HTML5
\lstdefinelanguage{HTML5}{
  language=html,
  sensitive=true,	
  alsoletter={<>=-},	
  morecomment=[s]{<!-}{-->},
  tag=[s],
  otherkeywords={
  % General
  >,
  % Standard tags
	<!DOCTYPE,
  </html, <html, <head, <title, </title, <style, </style, <link, </head, <meta, />,
	% body
	</body, <body,
	% Divs
	</div, <div, </div>, 
	% Paragraphs
	</p, <p, </p>,
	% scripts
	</script, <script,
  % More tags...
  <canvas, /canvas>, <svg, <rect, <animateTransform, </rect>, </svg>, <video, <source, <iframe, </iframe>, </video>, <image, </image>, </section, <section, <span, </span, <h1, </h1
  },
  ndkeywords={
  % General
  =,
  % HTML attributes
  charset=, src=, id=, width=, height=, style=, type=, rel=, href=, class=, data-onslide=, data-onstyle=,
  % SVG attributes
  fill=, attributeName=, begin=, dur=, from=, to=, poster=, controls=, x=, y=, repeatCount=, xlink:href=,
  % CSS properties
  margin:, padding:, background-image:, border:, top:, left:, position:, width:, height:,
	% CSS3 properties
  transform:, -moz-transform:, -webkit-transform:,
  animation:, -webkit-animation:,
  transition:,  transition-duration:, transition-property:, transition-timing-function:,
  }
}

\lstset{%
  % General design
  % backgroundcolor=\color{editorGray},
  basicstyle={\small\ttfamily},   
  frame=l,
  % line-numbers
  % xleftmargin={0.75cm},
  % numbers=left,
  % stepnumber=1,
  % firstnumber=1,
  % numberfirstline=true,	
  % Code design
  identifierstyle=\color{black},
  keywordstyle=\color{blue}\bfseries,
  ndkeywordstyle=\color{editorGreen}\bfseries,
  stringstyle=\color{editorOcher}\ttfamily,
  commentstyle=\color{darkgray}\ttfamily,
  % Code
  language=HTML5,
  alsolanguage=JavaScript,
  alsodigit={.:;},	
  tabsize=2,
  showtabs=false,
  showspaces=false,
  showstringspaces=false,
  extendedchars=true,
  breaklines=true,
  % German umlauts
  literate=%
         {á}{{\'a}}1
         {í}{{\'i}}1
         {é}{{\'e}}1
         {ú}{{\'u}}1
         {ó}{{\'o}}1
         {Á}{{\'A}}1
         {Í}{{\'I}}1
         {É}{{\'E}}1
         {Ú}{{\'U}}1
         {Ó}{{\'O}}1
         {Ň}{{\v{N}}}1
}

\newcommand{\R}{\mathbbm R}
\newcommand{\Rm}{\mathbbm{R}^{m}}
\newcommand{\Rn}{\mathbbm{R}^{n}}
\newcommand{\code}[1]{\texttt{#1}}
\newcommand{\codet}[1]{\texttt{\textbf{#1}}}
\newcommand{\htbl}[1]{\textsf{#1}}
\newcommand{\rec}[1]{ \tikz \node[rectangle,draw=red!25,thick,fill=red!20,align=center]{#1}; }
\newcommand{\fig}[2]{\includegraphics[width=#1\linewidth]{gfx//#2}} %width=\linewidth
\newcommand{\xcancel}[1]{\tikz[baseline=(X.base),opacity=.7]\node[cross out,draw=red,line width=.3ex](X){#1};}

\title{Beampress y Beampressk: herramientas web para mostrar y controlar diapositivas }
\author{Jorge Roberto Curbelo Fernández}
\advisor{M.Sc. Fernando Raúl Rodríguez Flores}
\degree{Licenciado en Ciencia de la Computación}
\faculty{Facultad de Matemática y Computación}
\date{Junio de 2015}
\logo{gfx/uhlogo.pdf}
\makenomenclature

\hyphenation{he-rra-mien-tas Py-thon dot-Value}

%%\flushleft
%%\newenvironment{vardesc}[1]{%
%%\settowidth{\parindent}{#1:\ }
%%\makebox[0pt][r]{#1:\ }}{}
%%\begin{displaymath}
%%a^2+b^2=c^2
%%\end{displaymath}
%%\begin{vardesc}{Where}$a$,
%%$b$ -- are adjunct to the right
%%angle of a right-angled triangle.
%%$c$ -- is the hypotenuse of
%%the triangle and feels lonely.
%%$d$ -- finally does not show up
%%here at all. Isn't that puzzling?
%%\end{vardesc}

%%%}}}


%%% Local Variables: 
%%% mode: latex
%%% TeX-master: "thesis"
%%% End: 
