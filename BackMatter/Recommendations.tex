%Recommendations
%!TEX root = ../thesis.tex
\begin{recomendations}

	Se recomienda reducir la sintaxis extendida para definir \textit{overlays}, con el propósito de disminuir el volumen de código de los atributos \texttt{data-onslide}.

	Se sugiere añadir a la sintaxis básica el uso de los comodines \texttt{.} y \texttt{+} (usados por Beamer) para que las definiciones de los overlays resulte más sencillo. 

	Se propone incorporar nuevas funciones de animación usando otras herramientas distintas a jQuery como \textit{GreenSock Animation Platform} (GSAP). 

	Se sugiere automatizar el proceso de añadir una presentación a Beampressk, ya que actualmente el usuario debe crear el directorio donde será ubicado el \textit{template} de la presentación y definir la locación de los recursos que utilizará como son los audios y los videos.

	

\end{recomendations}
