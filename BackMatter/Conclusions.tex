% Conclusions
%!TEX root = ../thesis.tex
\begin{conclusions}

	Beampress es un sistema para hacer diapositivas basadas en HTML5, CSS3 y JavaScript. Las ventajas de su implementación se explican a través de sus características, como el hecho de que las diapositivas se encuentren en ámbito web, lo cual determina que se estructuren con HTML proporcionando la capacidad de separar contenido de forma. Además dispone de múltiples animaciones y permite añadir nuevas funciones de animación a las opciones del plugin \textit{beampress.js}, dado su poder de extensibilidad.

	Beampressk, que es un servidor implementado en Flask, ofrece varias posibilidades. Permite el control de la presentación mediante una comunicación instantánea que logra utilizando Websocket entre la vista de control que define y la vista de presentación. También permite la comunicación con otros servicios web así como la manipulación de los contenidos de las diapositivas.

	Estas dos herramientas son particularmente útiles para crear y controlar presentaciones, como se demuestra en el Proyecto Delta, un espacio para la divulgación de la ciencia, la teconlogía y el humor, en cuyas apariciones públicas ambas han sido utilizadas. 

\end{conclusions}
